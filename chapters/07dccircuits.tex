\chapter{ไฟฟ้ากระแสตรง}
\section{การวิเคราะห์วงจร}

\subsection{กฎของ Kirchhoff}

เรามี ``กฎของ Ohm'' สำหรับแต่ละ \emph{passive component} (อุปกรณ์ที่ไม่สร้างพลังงาน) ดังนี้
\begin{lawbox}{ความสัมพันธ์ของ v และ i สำหรับ Passive Component}
    สำหรับตัวต้านทาน:
    \begin{equation}
        v = iR
    \end{equation}
    และสำหรับตัวเหนี่ยวนำ:
    \begin{equation}
        v = L\odv{i}{t} \label{endohm}
    \end{equation}
    สำหรับตัวเก็บประจุ:
    \begin{equation}
        i = C\odv{v}{t} \label{begohm}
    \end{equation}
\end{lawbox}
(ในบทนี้เราจะใช้ตัวอักษร $v$ และ $i$ ที่เป็นตัวพิมพ์เล็กเพื่อแทนความต่างศักย์และกระแสที่อาจขึ้นกับเวลา) ซึ่งสามารถนำมาใช้ในการวิเคราะห์วงจรได้ด้วยกฎของ Kirchhoff:

พิจารณาวงจร ณ จุด ๆ หนึ่ง ถ้าที่จุดนั้นไม่มีประจุสะสมอยู่เลยโดย \ref{contj} จะได้ว่า
\begin{ieqbox}{Kirchhoff's Current Law (KCL)}
    \sum_\txt{junction} i = 0\label{kcl}
\end{ieqbox}

โดยกฎนี้ใช้ในการวิเคราะห์วงจรแบบ\emph{โนด} (\emph{nodal analysis}) โดยเริ่มจากการตั้งศักย์ไฟฟ้าบนแต่ละโนดและกระแสที่ไหลเข้าและออกจากแต่ละโนด จากนั้นใช้ (\ref{kcl}) และ (\ref{begohm}) ถึง (\ref{endohm}) ในการเขียนทุกตัวแปรให้อยู่ในรูป $V$

ต่อมาพิจารณาวงจรที่ไม่มีการเปลี่ยนแปลงสนามแม่เหล็ก จะได้ว่า $\vb{E}$ เป็นสนามอนุรักษ์ ดังนั้น
\begin{ieqbox}{Kirchhoff's Voltage Law (KVL)}
    \sum_\txt{loop} v = 0\label{kvl}
\end{ieqbox}

กฎนี้ใช้ในการวิเคราะห์วงจรแบบ\emph{ลูป} (\emph{mesh analysis}) โดยเริ่มจากกำหนดกระแสที่วนอยู่ในแต่ละลูปที่กำหนดขึ้น จากนั้นใช้ (\ref{kvl}) และ (\ref{begohm}) ถึง (\ref{endohm}) ตั้งสมการตามจำนวนลูปที่กำหนดไว้เพื่อแก้หา $I$ ในแต่ละลูป

\subsection{การต่อตัวต้านทาน, ตัวเก็บประจุ, และตัวเหนี่ยวนำอย่างง่าย}

พิจารณาการต่อตัวต้านทาน $R_1$ และ $R_2$ แบบ\emph{อนุกรม} จะได้ว่ากระแส $i$ บนตัวต้านทาน $R_1$ จะเท่ากับ $I$ บนตัวต้านทาน $R_2$ ดังนั้น
\[ 
v_\txt{total}=v_1+v_2=iR_1+iR_2
\]
จึงได้ความต้านทานรวมเท่ากับ
\[ 
R_\txt{total}=R_1+R_2
\]
โดยเราสามารถทำแบบนี้ไปได้เรื่อย ๆ ด้วยตัวต้านทานกี่ตัวก็ได้ ดังนั้น
\begin{eqbox}{การต่อตัวต้านทานแบบอนุกรม}
    R_\txt{total}=R_1+R_2+\dots+R_n
\end{eqbox}
และพิจารณาการต่อตัวต้านทาน $R_1$ และ $R_2$ แบบ\emph{ขนาน} จะได้ว่าความต่างศักย์ของตัวต้านทานทั้งสองจะต้องเท่ากัน ดังนั้น
\[ 
i_\txt{total}=i_1+i_2=\frac{v}{R_1}+\frac{v}{R_2}
\]
จึงได้ความต้านทานรวมเท่ากับ
\[ 
\frac{1}{R_\txt{total}}=\frac{1}{R_1}+\frac{1}{R_2}
\]
โดยเช่นเดียวกับการต่อแบบอนุกรม เราสามารถทำแบบนี้ไปได้เรื่อย ๆ ด้วยตัวต้านทานกี่ตัวก็ได้ ดังนั้น
\begin{eqbox}{การต่อตัวต้านทานแบบขนาน}
    \frac{1}{R_\txt{total}}=\frac{1}{R_1}+\frac{1}{R_2}+\dots+\frac{1}{R_n}
\end{eqbox}

ต่อมาเช่นเดียวกับตัวต้านทาน พิจารณาการต่อตัวเก็บประจุ $C_1$ และ $C_2$ แบบอนุกรม จะได้ว่า $q$ บนตัวเก็บประจุ $C_1$ จะเท่ากับ $q$ บนตัวเก็บประจุ $C_2$ ดังนั้น
\[ 
v_\txt{total}=v_1+v_2=\frac{q}{C_1}+\frac{q}{C_2}
\]
เราสามารถทำแบบนี้ไปได้เรื่อย ๆ ด้วยตัวเก็บประจุกี่ตัวก็ได้ ดังนั้น
\begin{eqbox}{การต่อตัวเก็บประจุแบบอนุกรม}
    \frac{1}{C_\txt{total}}=\frac{1}{C_1}+\frac{1}{C_2}+\dots+\frac{1}{C_n}
\end{eqbox}
และพิจารณาการต่อตัวเก็บประจุ $C_1$ และ $C_2$ แบบขนาน จะได้ว่าความต่างศักย์คร่อมตัวเก็บประจุทั้งสองจะต้องเท่ากัน ดังนั้น
\[ 
Q_\txt{total}=Q_1+Q_2=C_1v+C_2v
\]
เราสามารถทำแบบนี้ไปได้เรื่อย ๆ ด้วยตัวเก็บประจุกี่ตัวก็ได้ ดังนั้น
\begin{eqbox}{การต่อตัวเก็บประจุแบบขนาน}
    C_\txt{total}=C_1+C_2+\dots+C_n
\end{eqbox}

สุดท้าย พิจารณาการต่อตัวเหนี่ยวนำ $L_1$ และ $L_2$ แบบอนุกรม จะได้ว่ากระแสที่ไหลผ่านตัวเหนี่ยวนำทั้งสองจะต้องเท่ากัน ดังนั้น
\[ 
v_\txt{total} = v_1 + v_2=L_1\odv{i}{t}+L_2\odv{i}{t}=L_1\odv{i}{t}+L_2\odv{i}{t}
\]
ก็จะได้
\begin{eqbox}{การต่อตัวเหนี่ยวนำแบบอนุกรม}
    L_\txt{total}=L_1+L_2+\dots+L_n
\end{eqbox}
และพิจารณาการต่อตัวเหนี่ยวนำ $L_1$ และ $L_2$ แบบขนาน จะได้ความต่างศักย์บนตัวเหนี่ยวนำทั้งสองเท่ากัน ดังนั้น
\[
\odv{i_\txt{total}}{t} = \odv{i_1}{t} + \odv{i_2}{t} = \frac{v}{L_1} + \frac{v}{L_2}
\]
ก็จะได้
\begin{eqbox}{การต่อตัวเหนี่ยวนำแบบขนาน}
    \frac{1}{L_\txt{total}}=\frac{1}{L_1}+\frac{1}{L_2}+\dots+\frac{1}{L_n}
\end{eqbox}

\section{วงจรอันดับหนึ่ง}

\subsection{อันดับของวงจร}

\begin{defbox}{อันดับของวงจร}
    \emph{อันดับของวงจร}คืออันดับของสมการเชิงอนุพันธ์ที่อธิบายวงจร เช่นวงจรไฟฟ้ากระแสตรงที่มีแค่แบตเตอรี่และตัวต้านทานไม่มีอนุพันธ์อะไรเลย จึงเป็นวงจรอันดับศูนย์
\end{defbox}

โดยวงจรอันดับหนึ่งได้แก่ วงจรที่มีตัวต้านทานและตัวเก็บประจุ (วงจร RC) และวงจรที่มีตัวต้านทานและตัวเหนี่ยวนำ (วงจร RL) และวงจรอันดับสองได้แก่วงจรที่มีตัวเก็บประจุและตัวเหนี่ยวนำ (วงจร LC และ RLC)

\subsection{วงจร RC}

วงจร RC เป็นวงจรอันดับหนึ่ง โดยจะยกตัวอย่างโจทย์การปล่อยประจุ (\emph{discharge}) จากตัวเก็บประจุ:

\begin{corbox}{ตัวอย่าง}
    จงหา $v(t)$ คร่อมตัวเก็บประจุของวงจรที่มีการต่อตัวเก็บประจุ $C$ และตัวต้านทาน $R$ แบบอนุกรม โดยที่ $C$ มีประจุเริ่มต้น $Q_0$
\end{corbox}

\begin{soln}
    ให้ $i_R$ และ $i_C$ คือกระแสที่ไหลออกจากจุด ๆ หนึ่งที่อยู่ฝั่งบวกของตัวเก็บประจุ จากนั้นใช้ KCL จะได้
    \begin{align*}
        i_R + i_C &= 0\\
        \frac{v}{R} + C\odv{v}{t'} &= 0\\
        -\frac{1}{RC}\odif{t'} &= \frac{1}{v} \odif{v}\\
        -\int_0^t \frac{1}{RC}\odif{t'} &= \int_{V_0}^{v(t)} \frac{1}{v} \odif{v}\\
        -\frac{1}{RC}t &= \log\ab(\frac{v(t)}{V_0})
    \end{align*}
    เนื่องจาก $V_0 = Q_0 / C$ ก็จะได้
    \[
    v(t) = \frac{Q_0}{C}e^{-\frac{1}{RC}t}
    \]
    โยเราจะเรียก $\tau \equiv RC$ ว่า\emph{ค่าคงที่เวลา} (\emph{time constant})
\end{soln}

พิจารณาวงจรที่มี $C$ ที่ \emph{steady state} (เมื่อจงจรเป็น steady current) ก็จะได้ว่า
\[
i_C = C\cancel{\odv{v}{t}} = 0 
\]
ดังนั้นจึงได้ว่า
\begin{lawbox}{ตัวเก็บประจุในวงจรไฟฟ้ากระแสตรงที่ Steady State}
    เมื่อ $t\to\infty$ จะสามารถมองได้ว่าตัวเก็บประจุ $C$ เปรียบเสมือนสายไฟขาด
\end{lawbox}

\subsection{วงจร RL}

วงจร RL เป็นวงจรอันดับหนึ่ง โดยจะยกตัวอย่างโจทย์การต่อแบตเตอรี่กับวงจรที่มี $L$:

\begin{corbox}{ตัวอย่าง}
    จงหา $i(t)$ และค่าคงที่เวลา $\tau$ ของการต่อแบตเตอรี่ที่มีแรงเคลื่อนไฟฟ้า $\emf$ ในวงจรที่มีการต่อตัวต้านทาน $R$ และตัวเหนี่ยวนำ $L$ แบบอนุกรม โดยที่ ณ เวลา $t = 0$ ไม่มีกระแสไหลอยู่เลย
\end{corbox}

\begin{soln}
    วนลูปที่มีกระแส $i(t)$ รอบวงจร จากนั้นใช้ KVL จะได้
    \begin{align*}
        v_R + v_L - \emf &= 0\\
        iR + L\odv{i}{t'} &= \emf\\
        -\frac{1}{L} \odif{t'} &= \frac{1}{iR-\emf} \odif{i}\\
        -\int_0^t \frac{1}{L} \odif{t'} &= \int_0^{i(t)} \frac{1}{iR-\emf}\odif{i}\\
        -\frac{R}{L}t &= \log\ab(\frac{\emf - Ri(t)}{\emf})
    \end{align*}
    ดังนั้นก็จะได้
    \[
    i(t) = \frac{\emf}{R}\ab(1-e^{-\frac{R}{L}t})
    \]
    และค่าคงที่เวลา $\tau = L / R$
\end{soln}

พิจารณาวงจรที่มี $L$ ที่ steady state ก็จะได้ว่า
\[
v = L\cancel{\odv{i}{t}} = 0 
\]
ดังนั้นจึงได้ว่า
\begin{lawbox}{ตัวเหนี่ยวนำในวงจรไฟฟ้ากระแสตรงที่ Steady State}
    เมื่อ $t\to\infty$ จะสามารถมองได้ว่าตัวเหนี่ยวนำ $L$ เปรียบเสมือนสายไฟเปล่า
\end{lawbox}

\subsection{วงจรอันดับหนึ่งในรูปทั่วไป}

โดยทั่วไปแล้วสมการเชิงอนุพันธ์ของวงจรอันดับหนึ่งจะอยู่ในรูป
\[
A\odv{i}{t} + Bi = C
\]
(หรือบางครั้งอาจติดอยู่ในรูป $v$) โดยเมื่อเราแก้สมการออกมาจะได้
\begin{ieqbox}{คำตอบทั่วไปของวงจรอันดับหนึ่ง}
    i(t) = \frac{C}{A} + \left(I_0 - \frac{C}{A} \right) e^{-\frac{B}{A}t}
\end{ieqbox}
ก็จะได้ว่าโดยทั่วไปแล้ว ค่าคงที่เวลา $\tau$ จะเท่ากับ
\begin{ieqbox}{ค่าคงที่เวลาโดยทั่วไป}
    \tau = \frac{A}{B}
\end{ieqbox}
เราจะเรียกช่วงแรกที่เกิดการปรับตัวของกระแสหรือความต่างศักย์อย่างรวดเร็ว (ก่อน steady state) ว่า \emph{transient response}

\section{วงจรอันดับสอง}

\subsection{วงจร RLC แบบอนุกรม}

พิจารณาการแก้สมการของวงจรที่เป็น RLC ที่ต่อแบบอนุกรมโดยไม่มีแบตเตอรี่ โดย KVL จะได้ (ให้กระแสไหลออกจากฝั่งลบของตัวเก็บประจุ)
\begin{align}
    v_R + v_L + v_C &= 0\notag\\
    \odv{}{t}\ab(v_R + v_L + v_C) &= 0\notag\\
    L\odv[2]{i}{t} + R\odv{i}{t} + \frac{1}{C}i &= 0\label{rlcseries}
\end{align}

เราสามารถแก้สมการนี้ได้โดยการใช้ราก $s_{1,2}$ ของ characteristic equation:
\[
x^2 + (R/L)x + 1/LC =0
\]
ก็จะได้คำตอบคือ
\begin{eqbox}{คำตอบของ Characteristic Equation ของ RLC แบบอนุกรม}
    s_{1,2} = -\frac{R}{2L}\pm\sqrt{\ab(\frac{R}{2L})^2 - \frac{1}{LC}}
\end{eqbox}

โดยเราจะนิยามราก $s_{1,2}$ ของสมการว่าเป็น\emph{ความถี่ธรรมชาติ}ของวงจร และก็จะนิยาม \emph{damping factor} ($\alpha$) และ\emph{ความถี่ resonant} ($\omega_0$) (\emph{undamped natural frequency}) ดังนี้:
\begin{defbox}{ Damping Factor และความถี่ Resonant}
    \begin{equation}
    s_{1,2} \equiv -\alpha \pm \sqrt{\alpha^2 - \omega_0^2}
    \end{equation}
\end{defbox}
หมายเหตุ: \emph{โดยเราจะใช้หน่วย $\unit{Np/s}$ สำหรับ $\alpha$ แต่จริง ๆ แล้วหน่วย $\unit{Np}$ (neper) นี้เป็นน่วยที่ไม่มีมิติเหมือนกับ $\unit{rad}$}

และเราจะนิยามอัตราส่วนระหว่าง $\alpha$ และ $\omega_0$ ว่า \emph{damping ratio} ($\zeta$) ซึ่งเป็นค่าที่บ่งบอกว่าระบบถูก \emph{damp} ไปแค่ไหน: 
\begin{defbox}{ Damping Ratio}
    \begin{equation}
        \zeta \equiv \frac{\alpha}{\omega_0}
    \end{equation}
\end{defbox}

ในที่นี้เราก็จะได้
\begin{eqbox}{Damping Factor, ความถี่ Resonant, และ Damping Ratio ของ RLC แบบอนุกรม}
    \alpha = \frac{R}{2L}\qquad\omega_0 = \frac{1}{\sqrt{LC}}\qquad\zeta = \frac{R}{2}\sqrt{\frac{C}{L}}
\end{eqbox}

เมื่อแก้สมการเชิงอนุพันธ์ ถ้าราก $s_{1,2}$ เป็นจำนวนจริง ($\zeta > 1$) จะได้ว่า
\begin{eqbox}{วงจร RLC อนุกรมแบบ Overdamped}
    i(t) = A_1e^{s_1 t} + A_2e^{s_2 t}\label{starthom}
\end{eqbox}
โดยเราจะเรียกว่าเป็นวงจร \emph{overdamped} ซึ่งจะมีกราฟเป็นการลดลงของกระแส

แต่ถ้าราก $s_{1,2}$ เป็นรากซ้ำ ($\zeta = 0$) โดยให้เป็น $s$ จะได้ว่า
\begin{eqbox}{วงจร RLC อนุกรมแบบ Critically Damped}
    i(t) = (A_2 + A_1t)e^{st}\label{midhom}
\end{eqbox}
โดยเราจะเรียกว่าเป็นวงจร \emph{critically damped} ซึ่งจะมีกราฟเป็นการลดลงของกระแสที่มากที่สุดที่ยังไม่เกิดการสั่นขึ้นลงของกราฟ (decay แบบยังไม่มี oscillatory behavior)

และสุดท้าย ถ้าราก $s_{1,2}$ ไม่เป็นจำนวนจริง ($\zeta < 1$) เราจะนิยาม\emph{ความถี่ damped} (\emph{damped natural frequency}):
\begin{defbox}{ความถี่ Damped}
    \begin{equation}
        \omega_d \equiv \sqrt{\omega_0^2 - \alpha^2}
    \end{equation}
\end{defbox}
และก็จะได้คำตอบของสมการว่า
\begin{eqbox}{วงจร RLC อนุกรมแบบ Underdamped}
    i(t) = e^{-\alpha t} \left( A_1 \cos(\omega_d t) + A_2 \sin(\omega_d t) \right)\label{endhom}
\end{eqbox}
โดยเราจะเรียกว่าเป็นวงจร \emph{underdamped} ซึ่งกราฟจะมีการสั่นขึ้นลงของกระแสโดยมีแอมพลิจูดลดลงเรื่อย ๆ (decay แบบ oscillatory)

ในกรณีที่เป็นวงจร LC ($R = 0$, $\zeta = 0$) จะได้ว่า $\alpha = 0$ และคำตอบของสมการอยู่ในกรณี underdamped โดยจะคำตอบเป็นดังนี้:
\begin{eqbox}{จงจร LC แบบอนุกรม}
    i(t) = A_1 \cos(\omega_d t) + A_2 \sin(\omega_d t)\label{finalhom}
\end{eqbox}
โดยเราอาจจะเรียกว่าเป็น RLC แบบ \emph{undamped} ซึ่งกราฟจะมีแค่การสั่นขึ้นลงแต่ไม่มีการลดลงของแอมพลิจูด (completely oscillatory)

\subsection{วงจร RLC แบบขนาน}

พิจารณาการแก้สมการของวงจรที่เป็น RLC ที่ต่อแบบขนานโดยไม่มีแบตเตอรี่ โดย KCL จะได้ (ให้กระแสไหลออกจากฝั่งลบของตัวเก็บประจุ)
\begin{align}
    i_R + i_L + i_C &= 0\notag\\
    \odv{}{t}\ab(i_R + i_L + i_C) &= 0\notag\\
    C\odv[2]{v}{t} + \frac{1}{R}\odv{v}{t} + \frac{1}{L}v &= 0\label{rlcparallel}
\end{align}

ก็จะได้ characteristic equation:
\[
x^2 + (1/RC)x + 1/LC = 0
\]
มีคำตอบ $s_{1,2}$ คือ
\begin{eqbox}{คำตอบของ Characteristic Equation ของ RLC แบบขนาน}
    s_{1,2} = -\frac{1}{2RC}\pm\sqrt{\ab(\frac{1}{2RC})^2 - \frac{1}{LC}}
\end{eqbox}
ก็จะได้ว่า
\begin{eqbox}{Damping Factor, ความถี่ Resonant, และ Damping Ratio ของ RLC แบบขนาน}
    \alpha = \frac{1}{2RC}\qquad\omega_0 = \frac{1}{\sqrt{LC}}\qquad\zeta = \frac{1}{2R}\sqrt{\frac{L}{C}}
\end{eqbox}

ต่อมาเมื่อแก้สมการเชิงอนุพันธ์ (เช่นเดียวกับในกรณีต่อแบบอนุกรมแต่ทีนี้หา $v$ แทน $i$) ก็จะได้
\begin{eqbox}{วงจร RLC ขนานแบบ Overdamped}
    v(t) = A_1e^{s_1 t} + A_2e^{s_2 t}
\end{eqbox}
ในกรณี overdamped,
\begin{eqbox}{วงจร RLC ขนานแบบ Critically Damped}
    v(t) = (A_2 + A_1t)e^{st}
\end{eqbox}
ในกรณี critically damped, และ
\begin{eqbox}{วงจร RLC ขนานแบบ Underdamped}
    v(t) = e^{-\alpha t} \left( A_1 \cos(\omega_d t) + A_2 \sin(\omega_d t) \right)
\end{eqbox}
ในกรณี underdamped

ส่วนในกรณีที่เป็นวงจร LC แบบขนาน ($R\to\infty$, $\zeta = 0$) หรือ RLC ขนานแบบ undamped จะได้ว่า $\alpha = 0$ และมีคำตอบคือ
\begin{eqbox}{จงจร LC แบบขนาน}
    v(t) = A_1 \cos(\omega_d t) + A_2 \sin(\omega_d t)
\end{eqbox}

\subsection{วงจรอันดับสองในรูปทั่วไป}

โดยทั่วไปแล้วสมการเชิงอนุพันธ์ของวงจรอันดับสองจะอยู่ในรูป
\[
A\odv[2]{i}{t} + B\odv{i}{t} + Ci = D
\]
(หรืออาจติดอยู่ในรูป $v$) โดยจะได้
\begin{ieqbox}{Damping Factor, ความถี่ Resonant, และ Damping Ratio โดยทั่วไป}
    \alpha = \frac{B}{2A}\qquad\omega_0 = \sqrt{\frac{C}{A}}\qquad\zeta = \frac{B}{2\sqrt{AC}}\label{resonant}
\end{ieqbox}
และจะได้ homogeneous solution เป็น (\ref{starthom}), (\ref{midhom}), (\ref{endhom}), และ (\ref{finalhom}) เมื่อ $\zeta > 1$, $\zeta = 1$, $0 < \zeta < 1$, และ $\zeta = 0$ ตามลำดับ และเมื่อแก้ particular solution ก็จะได้
\begin{ieqbox}{คำตอบทั่วไปของวงจรอันดับสอง}
    i(t) = i_h(t) + i_p(t) = i_h(t) + \frac{D}{C}
\end{ieqbox}

\section{กำลังไฟฟ้า}

\subsection{กำลังไฟฟ้าและประสิทธิภาพ}

จาก (\ref{jouleheat}) เราจะสามารถคำนวณกำลังไฟฟ้า $P$ ได้จาก
\begin{ieqbox}{กำลังไฟฟ้า}
    P = IV = I^2R = \frac{V^2}{R} = \odv{Q}{t} \label{power}
\end{ieqbox}
($Q$ คือความร้อนที่ได้จากวงจร) ซึ่งเราจะสามารถนำมาใช้คำนวณพลังงานที่เครื่องใช้ไฟฟ้านั้นถูกกระทำ โดยเนื่องจากพลังงานที่เครื่องใช้ไฟฟ้านี้ได้รับเป็นพลังงานความร้อน เราจึงต้องอาศัย \emph{heat engine} ในการเปลี่ยนความร้อนมาเป็นพลังงานที่เราสามารถนำไปใช้ประโยชน์ต่อได้ จึงนิยาม\emph{ประสิทธิภาพ} (\emph{efficiency}: $\eta$) ของเครื่องใช้ไฟฟ้า (heat engine) นี้ว่า
\begin{defbox}{ประสิทธิภาพ}
    \begin{equation}
        \eta \equiv \frac{W}{Q}
    \end{equation}
\end{defbox}
หมายเหตุ: \emph{$Q_C = Q - W$ เป็นความร้อนที่ถูกปล่อยทิ้งลงใน cold sink ซึ่งโดยกฎข้อที่สองของ thermodynamics จะได้ว่า $Q_C > 0$ ดังนั้น $\eta < 1$}
