\chapter{สัมพัทธภาพกับแม่เหล็กไฟฟ้า}

\section{ทฤษฎีสัมพัทธภาพพิเศษ}

\subsection{สัจพจน์ของ Einstein}

ในบทคลื่นแม่เหล็กไฟฟ้าเราจะเห็นว่าแสงเคลื่อนที่ด้วยความเร็ว $c = 1/\sqrt{\mu_0\eps_0}$ แต่ความเร็วนี้เทียบกับอะไรล่ะ? นั่นเป็นคำถามที่นักฟิสิกส์ในยุคก่อน Einstein ได้ถกเถียงกันและได้ตั้งทฤษฎีต่าง ๆ มามากมาย จนกระทั่ง Einstein ได้ลองคิดว่า จริง ๆ แล้วความเร็วแสงนี้อาจเท่ากันหมดไม่ว่าจะเป็นกรอบอ้างอิงเฉื่อยใด ๆ ก็ตาม และความคิดเรื่องสัมพัทธภาพแบบดั้งเดิม (Galileian relativity) ที่เป็นการนำความเร็วมาบวกกันและ simulteneity ของทุกกรอบอ้างอิงเหมือนกัน อาจไม่เป็นจริงก็ได้ โดย Einstein ได้ตั้งข้อสมมติไว้ดังนี้:
\begin{lawbox}{สัจพจน์ของ Einstein (Einstein's Postulates)}
    \begin{compactenum}
        \item กฎของฟิสิกส์สามารถใช้ได้บนกรอบอ้างอิงเฉื่อยทุกกรอบ
        \item ความเร็วแสงในสุญญากาศมีค่าเท่ากันสำหรับผู้สังเกตในทุก ๆ กรอบอ้างอิงเฉื่อย
    \end{compactenum}
\end{lawbox}

\subsection{เรขาคณิตของสัมพัทธภาพ}

จาก postulate แค่สองข้อนั้น เราสามารถได้ผลลัทธ์ต่าง ๆ มากมาย

ลองพิจารณารถบรรทุกคันหนึงที่เคลื่อนที่ด้วยอัตราเร็ว $v$ ไปทางขวาเทียบกับพื้นโดยมีแหล่งกำเนิดแสงอยู่ตรงกลาง เมื่อให้แหล่งกำเนิดปล่อยแสงมาในทุกทิศ ในมุมมองผู้สังเกตบนพื้นจะเห็นแสงกระทบกับผนังด้านซ้ายก่อนด้านขวา แต่ในมุมมองผู้สังเกตบนรถ แสงจะกระทบทั้งสองฝั่งพร้อม ๆ กัน จึงสรุปได้ว่า
\begin{lawbox}{ความไม่คงที่ของความเกิดขึ้นพร้อมกัน (Simulteneity)}
    เหตุการณ์สองเตุการณ์ที่เกิดขึ้นพร้อมกันในกรอบอ้างอิงหนึ่ง อาจเกิดขึ้นไม่พร้อมกันในอีกกรอบอ้างอิง
\end{lawbox}

ต่อมา พิจารณารถคันเดิมโดยสมมติว่าสูง $h$ แต่คราวนี้มาลองดูเวลาที่แสงใช้้ในการเดินทางจากเพดานลงมาถึงพื้น สำหรับผู้สังเกตบนรถ แสงเคลื่อนที่จากบนลงล่างเป็นเส้นตรงในแนวดิ่ง จะได้ว่าเวลา
\[
\Delta\bar{t} = \frac{h}{c}
\]
แต่สำหรับผู้สังเกตบนพื้น พื้นของรถเคลื่อนที่ออกไปแล้ว $v\Delta t$ ดังนั้นแสงจะต้องเคลื่อนที่แบบเอียง ๆ ก็จะได้
\begin{align*}
    c\Delta t &= \sqrt{h^2 + \ab(v\Delta t)^2}\\
    \Delta t &= \frac{1}{\sqrt{1 - v^2/c^2}}\frac{h}{c}
\end{align*}
ดังนั้นถ้าเรานิยาม
\begin{defbox}{ Lorentz Factor}
    \begin{equation}
        \gamma \equiv \frac{1}{\sqrt{1 - v^2/c^2}}
    \end{equation}
\end{defbox}
ก็จะได้ว่า
\begin{ieqbox}{การขยายขนาดของเวลา (Time Dilation)}
    \Delta\bar{t} = (1/\gamma)\Delta t\label{timedil}
\end{ieqbox}
หมายเหตุ: \emph{สมการนี้ใช้ได้เฉพาะผลต่างเวลาที่สองเหตุการณ์เกิดขึ้นที่ตำแหน่งเดียวกันเทียบกับกรอบอ้างอิงรถ โดยเหตุผลจะได้เห็นอีกที}

ต่อมา ให้รถนี้ยาว $\Delta\bar{x}$ และ $\Delta x$ สำหรับผู้สังเกตบนรถและบนพื้น ตามลำดับ เราจะแสดงว่าความยาวทั้งสองนี้ไม่เท่ากัน พิจารณาเวลาที่ใช้ในการที่ปล่อยแสงจากฝั่งซ้ายของรถให้เคลื่อนที่ไปทางขวาของรถที่มีกระจกวางไว้อยู่แล้วให้สะท้อนกลับมาที่ฝั้งซ้าย สำหรับผู้สังเกตบนรถ จะต้องใช้เวลา
\[
\Delta\bar{t} = 2\frac{\Delta\bar{x}}{c}
\]
สำหรับผู้สังเกตบนพื้น แบ่งเป็นเวลา $\Delta t_1$ และ $\Delta t_2$ คือเวลาที่แสงใช้เคลื่อนที่ไปและกลับ ตามลำดับ ดังนั้น
\[
\Delta t_1 = \frac{\Delta x + v\Delta t_1}{c}\qq{และ}\Delta t_2 = \frac{\Delta x - v\Delta t_2}{c}
\]
ก็จะได้
\[
\Delta t_1 = \frac{\Delta x}{c - v}\qq{และ}\Delta t_2 = \frac{\Delta x}{c + v}
\]
ดังนั้น
\[
\Delta t = 2\frac{\Delta x}{c}\frac{1}{1 - v^2/c^2}
\]
จาก (\ref{timedil}) จึงได้ว่า
\begin{ieqbox}{การหดตัวของความยาว (Length Contraction)}
    \Delta\bar{x} = \gamma\Delta x\label{lengthcont}
\end{ieqbox}

สุดท้าย พิจารณา