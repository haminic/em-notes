\chapter{สัมพัทธภาพกับแม่เหล็กไฟฟ้า}

\section{ทฤษฎีสัมพัทธภาพพิเศษ}

\subsection{สัจพจน์ของ Einstein}

ในบทคลื่นแม่เหล็กไฟฟ้าเราจะเห็นว่าแสงเคลื่อนที่ด้วยความเร็ว $c = 1/\sqrt{\mu_0\eps_0}$ แต่ความเร็วนี้เทียบกับอะไรล่ะ? นั่นเป็นคำถามที่นักฟิสิกส์ในยุคก่อน Einstein ได้ถกเถียงกันและได้ตั้งทฤษฎีต่าง ๆ มามากมาย จนกระทั่ง Einstein ได้ลองคิดว่า จริง ๆ แล้วความเร็วแสงนี้อาจเท่ากันหมดไม่ว่าจะเป็นกรอบอ้างอิงเฉื่อยใด ๆ ก็ตาม และความคิดเรื่องสัมพัทธภาพแบบดั้งเดิม (Galilean relativity) ที่เป็นการนำความเร็วมาบวกกันและ simulteneity ของทุกกรอบอ้างอิงเหมือนกัน อาจไม่เป็นจริงก็ได้ โดย Einstein ได้ตั้งข้อสมมติไว้ดังนี้:
\begin{lawbox}{สัจพจน์ของ Einstein (Einstein's Postulates)}
    \begin{compactenum}
        \item กฎของฟิสิกส์สามารถใช้ได้บนกรอบอ้างอิงเฉื่อยทุกกรอบ
        \item ความเร็วแสงในสุญญากาศมีค่าเท่ากันสำหรับผู้สังเกตในทุก ๆ กรอบอ้างอิงเฉื่อย
    \end{compactenum}
\end{lawbox}

\subsection{เรขาคณิตของสัมพัทธภาพ}

จาก postulate แค่สองข้อนั้น เราสามารถได้ผลลัทธ์ต่าง ๆ มากมาย

ลองพิจารณารถบรรทุกคันหนึงที่เคลื่อนที่ด้วยอัตราเร็ว $v$ ไปทางขวาเทียบกับพื้นโดยมีแหล่งกำเนิดแสงอยู่ตรงกลาง เมื่อให้แหล่งกำเนิดปล่อยแสงมาในทุกทิศ ในมุมมองผู้สังเกตบนพื้นจะเห็นแสงกระทบกับผนังด้านซ้ายก่อนด้านขวา แต่ในมุมมองผู้สังเกตบนรถ แสงจะกระทบทั้งสองฝั่งพร้อม ๆ กัน จึงสรุปได้ว่า
\begin{lawbox}{ความไม่คงที่ของความเกิดขึ้นพร้อมกัน (Simulteneity)}
    เหตุการณ์สองเตุการณ์ที่เกิดขึ้นพร้อมกันในกรอบอ้างอิงหนึ่ง อาจเกิดขึ้นไม่พร้อมกันในอีกกรอบอ้างอิง
\end{lawbox}

ต่อมา พิจารณารถคันเดิมโดยสมมติว่าสูง $h$ แต่คราวนี้มาลองดูเวลาที่แสงใช้้ในการเดินทางจากเพดานลงมาถึงพื้น สำหรับผู้สังเกตบนรถ แสงเคลื่อนที่จากบนลงล่างเป็นเส้นตรงในแนวดิ่ง จะได้ว่าเวลา
\[
\Delta\bar{t} = \frac{h}{c}
\]
แต่สำหรับผู้สังเกตบนพื้น พื้นของรถเคลื่อนที่ออกไปแล้ว $v\Delta t$ ดังนั้นแสงจะต้องเคลื่อนที่แบบเอียง ๆ ก็จะได้
\begin{align*}
    c\Delta t &= \sqrt{h^2 + \ab(v\Delta t)^2}\\
    \Delta t &= \frac{1}{\sqrt{1 - v^2/c^2}}\frac{h}{c}
\end{align*}
ดังนั้นถ้าเรานิยาม
\begin{defbox}{ Lorentz Factor}
    \begin{equation}
        \gamma \equiv \frac{1}{\sqrt{1 - v^2/c^2}}
    \end{equation}
\end{defbox}
ก็จะได้ว่า
\begin{ieqbox}{การขยายขนาดของเวลา (Time Dilation)}
    \Delta\bar{t} = (1/\gamma)\Delta t\label{timedil}
\end{ieqbox}
หมายเหตุ: \emph{สมการนี้ใช้ได้เฉพาะผลต่างเวลาที่สองเหตุการณ์เกิดขึ้นที่ตำแหน่งเดียวกันเทียบกับกรอบอ้างอิงรถ โดยเหตุผลจะได้เห็นอีกที}

ต่อมา ให้รถนี้ยาว $\Delta\bar{x}$ และ $\Delta x$ สำหรับผู้สังเกตบนรถและบนพื้น ตามลำดับ เราจะแสดงว่าความยาวทั้งสองนี้ไม่เท่ากัน พิจารณาเวลาที่ใช้ในการที่ปล่อยแสงจากฝั่งซ้ายของรถให้เคลื่อนที่ไปทางขวาของรถที่มีกระจกวางไว้อยู่แล้วให้สะท้อนกลับมาที่ฝั้งซ้าย สำหรับผู้สังเกตบนรถ จะต้องใช้เวลา
\[
\Delta\bar{t} = 2\frac{\Delta\bar{x}}{c}
\]
สำหรับผู้สังเกตบนพื้น แบ่งเป็นเวลา $\Delta t_1$ และ $\Delta t_2$ คือเวลาที่แสงใช้เคลื่อนที่ไปและกลับ ตามลำดับ ดังนั้น
\[
\Delta t_1 = \frac{\Delta x + v\Delta t_1}{c}\qq{และ}\Delta t_2 = \frac{\Delta x - v\Delta t_2}{c}
\]
ก็จะได้
\[
\Delta t_1 = \frac{\Delta x}{c - v}\qq{และ}\Delta t_2 = \frac{\Delta x}{c + v}
\]
ดังนั้น
\[
\Delta t = 2\frac{\Delta x}{c}\frac{1}{1 - v^2/c^2}
\]
จาก (\ref{timedil}) จึงได้ว่า
\begin{ieqbox}{การหดตัวของความยาว (Length Contraction)}
    \Delta\bar{x} = \gamma\Delta x\label{lengthcont}
\end{ieqbox}

สุดท้าย สมมติข้าง ๆ ถนนมีกำแพงที่มีแถบสีน้ำเงินถูกทาไว้ $1$ เมตรเหนือถนนสำหรับผู้สังเกตบนพื้น สมมติมีคน ๆ หนึ่งอยู่บนรถ ถ้าคนนั้นโผล่หัวออกมาจากหน้าต่างแล้วใช้พู่กันทาสีแดงบนกำแพงเหนือถนน $1$ เมตรสำหรับเขา ถ้าสรุปแล้วแถบสีแดงไม่ได้อยู่ที่ตำแหน่งเดียวกันกับแถบสีน้ำเงินจะเกิดข้อขัดแย้ง เพราะโดยสัจพจน์ของ Einstein ในมุมมองผู้สังเกตบนพื้นจะได้ผลลัพธ์ตรงกันข้าม จึงสรุปได้ว่า
\begin{lawbox}{การแปลงของมิติที่ตั้งฉากกับความเร็ว}
    มิติที่ตั้งฉากกับความเร็วจะไม่เกิดการหดหรือขยาย
\end{lawbox}

\subsection{การแปลง Lorentz}

เราจะนิยาม\emph{เหตุการณ์} (\emph{event}) คือชุดของตำแหน่ง $(x,y,z)$ และเวลา $t$ โดยสัมพัทธภาพแบบ Galilean การแปลงพิกัดของเหตุการณ์ $E(x,y,z,t)$ จากกรอบอ้างอิง $\sur$ เป็น $E(\bar{x}, \bar{y}, \bar{z}, \bar{t})$ ในกรอบ $\bar{\sur}$ ซึ่งเคลื่อนที่ด้วยความเร็ว $v\vu{x}$ เทียบ $\sur$ สามารถทำได้โดย
\begin{equation}
    \begin{rcases}
        \bar{x} = x - vt\hspace{5pt}\\
        \bar{y} = y\\
        \bar{z} = z\\
        \bar{t} = t
    \end{rcases}\label{galileantrans}
\end{equation}
ซึ่งจะเรียกว่า\emph{การแปลง Galilean} (\emph{Galilean transformation})

แต่ถ้าเป็นการแปลงแบบสัมพัทธภาพพิเศษ เราจะได้ว่า
\[
x = d + vt
\]
เมื่อ $d$ คือระยะจาก $\bar{\org}$ ไปยัง $\bar{A}$ (ซึ่งคือจุดในแนวแกน $\bar{x}$ ที่เกิด $E$) ณ เวลา $t$ (ซึ่งเป็นระยะที่วัดในกรอบ $\sur$) โดยถ้าเป็นสัมพัทธภาพ Galilean เราจะได้พจน์ $d$ นี้ก็คือ $\bar{x}$ และได้ดัง (\ref{galileantrans}) แต่จาก (\ref{lengthcont}) เรารู้แล้วว่าจริง ๆ แล้ว
\[
d = \frac{1}{\gamma}\bar{x}
\]
ดังนั้น
\begin{equation}
    \bar{x} = \gamma\ab(x - vt)\tag{$\diamond$1}\label{lorentz1}
\end{equation}
แต่ถ้าเราคิดในทางกลับกันโดยวัด $\bar{d}$ ในกรอบ $\bar{\sur}$ จะได้ว่า
\begin{equation}
    x = \gamma\ab(\bar{x} + v\bar{t})\tag{$\diamond$2}\label{lorentz2}
\end{equation}
จาก (\ref{lorentz1}) และ (\ref{lorentz2}) จะได้แก้ว่า
\begin{equation}
    \bar{t} = \gamma\ab(t - \frac{v}{c^2}x)\tag{$\diamond$3}\label{lorentz3}
\end{equation}
ดังนั้นโดยสมการ (\ref{lorentz1}) และ (\ref{lorentz3}) เราก็จะได้การแปลงที่สมบูรณ์ ซึ่งเรียกว่าเป็น\emph{การแปลง Lorentz} (\emph{Lorentz transformation}) ดังนี้
\begin{ieqbox}{การแปลง Lorentz (Lorentz Transformation)}
    \begin{rcases}
        \bar{x} = \gamma(x - vt)\hspace{5pt}\\
        \bar{y} = y\\
        \bar{z} = z\\
        \bar{t} = \gamma\ab(t - \dfrac{v}{c^2}x)
    \end{rcases}\label{lorentztrans}
\end{ieqbox}

พิจารณาอนุภาคหนึ่งที่เคลื่อนที่ด้วยอัตราเร็ว $u$ ในกรอบ $\sur$ จะได้ว่า
\[
u = \odv{x}{t}
\]
แต่ใน $\sur'$ จาก (\ref{lorentztrans}) จะได้ว่า
\[
\odif{\bar{x}} = \gamma(\odif{x} - v\odif{t})
\]
และ
\[
\odif{\bar{t}} = \gamma\ab(\odif{t} - \frac{v}{c^2}\odif{x})
\]
ก็จะได้อัตราเร็วของอนุภาคนี้ใน $\sur'$ เท่ากับ
\[
\bar{u} = \odv{\bar{x}}{\bar{t}} = \frac{\gamma(\odif{x} - v\odif{t})}{\gamma\ab(\odif{t} - v/c^2\odif{x})} = \frac{u - v}{1 - uv/c^2}
\]
ได้เป็นสมการรวมความเร็วสัมพัทธ์แบบสัมพัทธภาพพิเศษ
\begin{eqbox}{กฎการรวมความเร็วของ Einstein}
    v_{AC} = \frac{v_{AB} + v_{BC}}{1 + v_{AB}v_{BC}/c^2}
\end{eqbox} 

\section{ปริภูมิ-เวลา}

\subsection{โฟร์เวกเตอร์}

การแปลง Lorentz สามารถเขียนได้อยู่ในรูปที่อ่านง่ายขึ้นถ้าเรานิยาม
\[
x^0\equiv ct \qq*{,}\beta\equiv\frac{v}{c}
\]
และ $x^1\equiv x$, $x^2\equiv y$, $x^3\equiv z$ จะได้
\[
\begin{rcases}
    \bar{x}^0 = \gamma(x^0 - \beta x^1)\hspace{5pt}\\
    \bar{x}^1 = \gamma(x^1 - \beta x^0)\\
    \bar{x}^2 = x^2\\
    \bar{x}^3 = x^3
\end{rcases}
\]
หรือก็คือ
\begin{eqbox}{การแปลง Lorentz ในรูปเมทริกซ์}
    \begin{bmatrix} 
        \bar{x}^0\\ 
        \bar{x}^1\\ 
        \bar{x}^2\\ 
        \bar{x}^3 
    \end{bmatrix} = 
    \begin{bmatrix} 
        \gamma& -\gamma\beta& 0& 0\\
        -\gamma\beta& \gamma& 0& 0\\
        0& 0& 1& 0\\
        0& 0& 0& 1
    \end{bmatrix}
    \begin{bmatrix}
        x^0\\
        x^1\\
        x^2\\
        x^3
    \end{bmatrix}
\end{eqbox}
ถ้าเรานิยามเมทริกซ์ตรงกลางว่าเป็น $\Lambda$ จะได้ว่า
\begin{eqbox}{การแปลง Lorentz ในรูป Einstein Summation}
    \bar{x}^\mu = \Lambda^\mu_\nu x^\nu
\end{eqbox}
โดยการคำนวณใน Einstein summation ให้รวมทุก ๆ ค่าที่เป็นไปได้ของทุก index ที่ปรากฎอยู่สองตัวต่อพจน์ เราจะเรียกชุดของตัวเลขสี่ตัวที่แปลงในแบบเดียวกับ $(x^0, x^1, x^2, x^3)$ ว่า\emph{โฟร์เวกเตอร์} (\emph{4-vector})