\chapter{ไฟฟ้ากระแสสลับ}

\section{นิยามของไฟฟ้ากระแสสลับ}

\subsection{แหล่งกำเนิดไฟฟ้ากระแสสลับ}

วงจรทั้งหมดที่เราเจอมาเรียกว่า\emph{วงจรไฟฟ้ากระแสตรง} (\emph{direct current} หรือ \emph{DC}) ซึ่งเกิดจากแหล่งกำเนิดที่ให้ความต่างศักย์คงที่ (หรือกระแสคงที่) ตลอดเวลา แต่วงจรไฟฟ้าที่เราใช้ในบ้านจะเป็น\emph{วงจรไฟฟ้ากระแสสลับ} (\emph{alternating current} หรือ \emph{AC}) ซึ่งเป็นวงจรที่แหล่งกำเนิดมี emf สลับทิศไปมาเป็นฟังก์ชันคาบ โดยไฟฟ้ากระแสสลับที่เราจะมาดูกันเราจะสมมติมาเป็นไฟฟ้ากระแสสลับรูปไซน์ (sinusoidal) ก็คือ
\begin{eqbox}{ไฟฟ้ากระแสสลับ}
    (i\txt{\ or\ }v)(t) = (I\txt{\ or\ }V)_m\cos \omega t
\end{eqbox}
เมื่อ $I_m$ และ $V_m$ คือ\emph{แอมพลิจูด}ของกราฟไซน์และ $\omega$ เรียกว่า\emph{ความถี่เชิงมุม} โดย argument ที่อยู่ในฟังก์ชันเราจะเรียกว่า\emph{เฟส} ณ ขณะนั้น และเราจะนิยาม\emph{คาบ} ($T$) และ\emph{ความถี่} ($f$) ตามปกติ:
\begin{defbox}{คาบและความถี่}
    \begin{equation}
        T\equiv\frac{2\pi}{\omega}\qq{และ}f\equiv\frac{1}{T} = \frac{\omega}{2\pi}
    \end{equation}
\end{defbox}

โดยยกตัวอย่างแหล่งกำเนิดไฟฟ้ากระแสสลับเช่น ลองพิจารณาสายไฟที่หมุนรอบแกน $x$ และมีสนามแม่เหล็กสม่ำเสมอขนาด $B$ ชี้ในแกน $+y$ ก็จะได้ว่าถ้าเราหมุนลูปสายไฟนี้ด้วยอัตราเร็วเชิงมุม $\omega$ แล้วก็จะได้ว่า
\[
\Phi_B = BA\cos\omega t
\]
ดังนั้น
\[
\emf = -\odv{\Phi_B}{t} = BA\omega \sin \omega t
\]
\begin{eqbox}{แหล่งกำเนิดไฟฟ้ากระแสสลับ}
    \emf = BA\omega\sin\omega t = BA\omega\cos\ab(\omega t + \pi/2)
\end{eqbox}

\subsection{ตัวต้านทาน, ตัวเก็บประจุ, และตัวเหนี่ยวนำ}

ต่อมาเรามาพิจารณาอุปกรณ์ที่เราคุ้นเคยกัน เริ่มจากตัวต้านทานเมื่อให้กระแสที่ไหลผ่าน $i(t) = I_m\cos\omega t$ เนื่องจาก $v = iR$ จะได้ว่า
\[
v(t) = Ri(t) = I_mR\cos\omega t = V_m\cos\omega t
\]
ดังนั้น
\begin{lawbox}{เฟสของ $v$ และ $i$ บนตัวต้านทาน}
    เฟสของ $v$ และ $i$ จะ\emph{ตรงกัน}บนตัวต้านทาน $R$ โดยที่
    \begin{equation}
        V_m = RI_m\label{resac}
    \end{equation}
\end{lawbox}

พิจารณาตัวเก็บประจุเมื่อให้ $v(t) = V_m\cos\omega t$ จะได้ว่า
\[
i(t) = C\odv{v}{t} = CV_m\odv{}{t}\cos\omega t = -\omega CV_m\sin\omega t = I_m\cos\ab(\omega t + \pi/2)
\]
ดังนั้น
\begin{lawbox}{เฟสของ $v$ และ $i$ บนตัวเก็บประจุ}
    เฟสของ $i$ จะนำ $v$ อยู่ $\pi/2$ บนตัวเก็บประจุ $C$ โดยที่
    \begin{equation}
        V_m = \ab(\frac{1}{\omega C})I_m\label{capac}
    \end{equation}
\end{lawbox}

สุดท้าย พิจารณาตัวเหนี่ยวนำเมื่อให้ $i(t) = I_m\cos\omega t$ จะได้ว่า
\[
v(t) = L\odv{i}{t} = LI_m\odv{}{t}\cos\omega t = -\omega LI_m\sin\omega t = V_m\cos\ab(\omega t + \pi/2)
\]
ดังนั้น
\begin{lawbox}{เฟสของ $v$ และ $i$ บนตัวเหนี่ยวนำ}
    เฟสของ $v$ จะนำ $i$ อยู่ $\pi/2$ บนตัวเหนี่ยวนำ $L$ โดยที่
    \begin{equation}
        V_m = \ab(\omega L)I_m\label{indac}
    \end{equation}
\end{lawbox}

สังเกตว่าสมการ (\ref{resac}) ถึง (\ref{indac}) หน้าตาคล้ายกับกฎของ Ohm เราจึงนิยามค่าความต้านทานเสมือนของตัวเก็บประจุและตัวเหนี่ยวนำว่า \emph{รีแอคแตนซ์เชิงประจุ}และ\emph{รีแอคแตนซ์เชิงเหนี่ยวนำ} (\emph{capacitive}/\emph{inductive reactance}: $X_C$ และ $X_L$) ดังนี้:
\begin{defbox}{ Reactance}
    \begin{equation}
        X_C \equiv \frac{1}{\omega C}\qq{และ}X_L \equiv \omega L
    \end{equation}
\end{defbox}

\section{เฟสเซอร์}

\subsection{Frequency Domain}

เราสามารถแทนฟังก์ชันไซน์ด้วยส่วนจริงของฟังก์ชัน exponential เชิงซ้อน:
\[
a(t) = A_m\cos\ab(\omega t + \phi) = \Re(A_me^{j\phi}\cex) \equiv \Re(A\cex)
\]
(ในการวิเคราะห์วงจรเราจะใช้ $j$ แทนหน่วยจินตภาพเพราะ $i$ ซ้ำกับกระแส) โดยจำนวนเชิงซ้อน $A$ เรียกว่าเป็นรูปใน\emph{โดเมนเฟสเซอร์}/\emph{ความถี่} (\emph{phasor}/\emph{frequency domain}) ของ $a(t)$ (โดยจะเรียก $a(t)$ ว่ารูป \emph{time domain}) โดยเราจะแทน $A = A_me^{j\phi}$ ด้วยสัญลักษณ์
\begin{defbox}{สัญลักษณ์แทนจำนวนเชิงซ้อนในการวิเคราะห์วงจร}
    \begin{equation}
        A_m\phase{\phi} \equiv A_me^{j\phi} = A 
    \end{equation}
\end{defbox}

\subsection{Impedance และ Admittance}

ต่อมาเรามาดู ``กฎของ Ohm" ของแต่ละอุปกรณ์ใน frequency domain, จาก (\ref{resac}) ถึง (\ref{indac}) ก็จะไดh
\begin{lawbox}{อุปกรณ์ต่าง ๆ ใน Frequency Domain}
    สำหรับตัวต้านทาน $R$:
    \begin{equation}
        V = RI
    \end{equation}
    สำหรับตัวเก็บประจุ $C$:
    \begin{equation}
        V = \ab(\frac{1}{j\omega C})I
    \end{equation}
    สำหรับตัวเหนี่ยวนำ $L$:
    \begin{equation}
        V = \ab(j\omega L)I
    \end{equation}
\end{lawbox}
เราจึงจะนิยามค่า\emph{อิมพีแดนซ์} (\emph{impedance}: $Z$) และ\emph{แอดมิตแทนซ์} (\emph{admittance}: $Y$) ว่า
\begin{defbox}{ Impedance และ Admittance}
    \begin{equation}
        Z\equiv \frac{V}{I}\qq{และ} Y\equiv\frac{1}{Z}=\frac{I}{V}
    \end{equation}
\end{defbox}

โดยที่ในส่วนจริงและส่วนจินตภาพของ impedance จะเป็นส่วนความต้านทาน (resistance: $R$) และส่วน reactance ($X$) ตามลำดับ:
\begin{eqbox}{ส่วนจริงและส่วนจินตภาพของ impedance}
    \Re(Z) = R\qq{และ}\Im(Z) = X\qq{หรือก็คือ}Z = R + jX
\end{eqbox}
และส่วนจริงและส่วนจินตภาพของ admittance จะเป็นส่วนความนำไฟฟ้า (conductance: $G$) และส่วน susceptance ($B$) ตามลำดับ:
\begin{eqbox}{ส่วนจริงและส่วนจินตภาพของ impedance}
    \Re(Y) = G\qq{และ}\Im(Y) = B\qq{หรือก็คือ}Y = G + jB
\end{eqbox}

และเราจะได้ impedance ของอุปกรณ์ต่าง ๆ:
\begin{eqbox}{Impedance ของอุปกรณ์ต่าง ๆ}
    Z_R = R\quad\quad Z_L = j\omega L \quad\quad Z_C = \frac{1}{j\omega C}
\end{eqbox}
admittance ของอุปกรณ์ต่าง ๆ:
\begin{eqbox}{Admittance ของอุปกรณ์ต่าง ๆ}
    Y_R = \frac{1}{R}\quad\quad Y_L = \frac{1}{j\omega L} \quad\quad Y_C = j\omega C
\end{eqbox}



\section{การวิเคราะห์วงจรใน Frequency Domain}

\subsection{กฎของ Kirchhoff ใน Frequency Domain}

เนื่องจากฟังก์ชัน $\Re$ และ $\Im$ มีสมบัติเชิงเส้น การแทนฟังก์ชันรูปไซน์ใน frequency domain จึงมีประโยชน์ต่อการคำนวณ เพราะกฎที่สำคัญต่าง ๆ ใน time domain ยังคงเป็นจริงใน frequency domain โดยที่สำคัญสุดเลยก็คือกฎของ Kirchhoff:
\begin{ieqbox}{KCL ใน Frequency Domain}
    \sum I_\txt{in} = \sum I_\txt{out}
\end{ieqbox}
และ
\begin{ieqbox}{KVL ใน Frequency Domain}
    \sum_\txt{loop} V = 0
\end{ieqbox}

\subsection{การรวม Impedance}

พิจารณาการต่อวงจรแบบอนุกรม โดยมีอุปกรณ์อยู่สองชิ้นที่มี impedance $Z_1$ และ $Z_2$ จะได้ว่ากระแส $I$ ที่ไหลผ่านอุปกรณ์ทั้งสองต้องเท่ากัน ดังนั้น
\begin{align*}
    v_\txt{total}(t) &= \Re(V_1\cex) + \Re(V_2\cex)\\
    &= \Re(IZ_1\cex) + \Re(IZ_2\cex)\\
    &= \Re\ab(I(Z_1+Z_2)\cex) 
\end{align*}
ดังนั้น $V_\txt{total} = I(Z_1+Z_2)$ แล้วก็จะได้
\[
Z_\txt{total} = Z_1 + Z_2
\]
ขยายเป็น $n$ อุปกรณ์จะได้
\begin{ieqbox}{การรวม Impedance แบบอนุกรม}
    Z_\txt{total} = Z_1 + Z_2 + \dots Z_n
\end{ieqbox}

ต่อมาพิจารณาการต่อวงจรแบบขนาน โดยมีอุปกรณ์สองชิ้นเดิม จะได้ว่าความต่างศักย์ $V$ คร่อมอุปกรณ์ทั้งสองต้องเท่ากัน ดังนั้น
\begin{align*}
    i_\txt{total}(t) &= \Re(I_1\cex) + \Re(I_2\cex)\\
    &= \Re\ab(\frac{V}{Z_1}\cex) + \Re\ab(\frac{V}{Z_2}\cex)\\
    &= \Re\ab(V\ab(\frac{1}{Z_1} + \frac{1}{Z_2})\cex)
\end{align*}
ดังนั้น $I_\txt{total} = V\ab(\frac{1}{Z_1} + \frac{1}{Z_2})$ และจะได้
\[
Y_\txt{total} = \frac{1}{Z_\txt{total}} = \frac{1}{Z_1} + \frac{1}{Z_2}
\]
ขยายเป็น $n$ อุปกรณ์จะได้
\begin{ieqbox}{การรวม Impedance แบบขนาน}
    \frac{1}{Z_\txt{total}} = \frac{1}{Z_1} + \frac{1}{Z_2} + \dots + \frac{1}{Z_n}
\end{ieqbox}
หรือ
\begin{eqbox}{การรวม Admittance แบบขนาน}
    Y_\txt{total} = Y_1 + Y_2 + \dots Y_n
\end{eqbox}

\subsection{Resonance ในวงจรไฟฟ้ากระแสสลับ}

จากสมการ

\section{กำลังไฟฟ้าและค่ายังผล}