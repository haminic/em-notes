\chapter{แม่เหล็กสถิต}
\section{กฎแรง Lorentz}
\subsection{แรงแม่เหล็ก}

\begin{lawbox}{แรง Lorentz}
    ประจุ $Q$ ที่เคลื่อนที่ด้วยความเร็ว $\vv$ ในสนามแม่เหล็ก $\vB$ จะถูกแรงแม่เหล็กกระทำดังนี้:
    \begin{equation}
        \vF_\txt{mag} = Q\ab(\vv\times\vB)
    \end{equation}
    โดยถ้ามีทั้งสนามไฟฟ้าและแม่เหล็ก:
    \begin{equation}
        \vF = Q\ab\big(\vE + \ab(\vv\times\vB))
    \end{equation}
\end{lawbox}
การเคลื่อนที่ใน $\vB$ สม่ำเสมอที่น่าสนใจมีดังนี้:
\begin{enumerate}
    \item ถ้าประจุ $Q$ เคลื่อนที่ด้วยความเร็ว $\vv$ ในสนาม $\vB$ เพียงอย่างเดียว ส่วนของ $\vv$
\end{enumerate}