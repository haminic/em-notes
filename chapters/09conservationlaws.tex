\chapter{กฎอนุรักษ์}

\section{พลังงาน}

\subsection{ทฤษฎีบทของ Poynting}

พิจารณางานที่สนามแม่เหล็กไฟฟ้ากระทำกับประจุหนึ่ง:
\[
\odif{W} = q\ab(\vb{E}+\vb{v}\times\vb{B})\cdot\odif{\vbs{\ell}} = q\ab(\vb{E}+\vb{v}\times\vb{B})\cdot \vb{v}\odif{t} = \vb{E}\cdot q\vb{v}\odif{t}
\]
ดังนั้นกำลังเท่ากับ
\begin{eqnobox}
    P = \odv{W}{t} = \int_\vol\ab(\vb{E}\cdot\vb{J})\odif{\tau}\label{poyntingone}
\end{eqnobox}
จาก
\[
\vb{J} = \frac{1}{\mu_0}\ab(\gd\times\vb{B}) - \eps_0\pdv{\vb{E}}{t}
\]
เมื่อนำไปแทนใน $\vb{E}\cdot\vb{J}$ แล้วใช้ product rule ก็จะได้
\begin{align*}
    \vb{E}\cdot\vb{J} &= \vb{E}\cdot\ab(\frac{1}{\mu_0}\ab(\gd\times\vb{B}) - \eps_0\pdv{\vb{E}}{t})\\
    &= -\frac{1}{\mu_0}\gd\cdot\ab(\vb{E}\times\vb{B}) + \frac{1}{\mu_0}\vb{B}\cdot\ab(\gd\times\vb{E}) - \eps_0\vb{E}\cdot\pdv{\vb{E}}{t}\\
    &= -\frac{1}{\mu_0}\vb{B}\cdot\pdv{\vb{B}}{t} - \eps_0\vb{E}\cdot\pdv{\vb{E}}{t} -\frac{1}{\mu_0}\gd\cdot\ab(\vb{E}\times\vb{B})
\end{align*}
เนื่องจาก $\pdv{}{t}A^2 = \pdv{}{t}\vb{A}\cdot\vb{A} = 2\vb{A}\cdot\pdv{\vb{A}}{t}$ ดังนั้น
\begin{eqnobox}
    \vb{E}\cdot\vb{J} = -\frac{1}{2}\pdv{}{t}\ab(\eps_0E^2 + \frac{1}{\mu_0}B^2) - \frac{1}{\mu_0}\gd\cdot\ab(\vb{E}\times\vb{B})\label{poyntingtwo}
\end{eqnobox}
เราจึงนิยาม \emph{Poynting vector}:
\begin{defbox}{ Poynting Vector}
    \begin{equation*}
        \vb{S}\equiv\frac{1}{\mu_0}\ab(\vb{E}\times\vb{B})
    \end{equation*}
\end{defbox}
นำ (\ref{poyntingtwo}) แทนใน (\ref{poyntingone}) และใช้ divergence theorem จะได้

\begin{ieqbox}{ทฤษฎีบทของ Poynting}
    P = \odv{W}{t} = -\odv{}{t}\int_\vol u\odif{\tau} - \oint_{\del\vol}\vb{S}\cdot\odif{\vb{a}}    
\end{ieqbox}
และถ้าสมมติว่าระบบไม่โดนงานมากระทำเลย (แต่โดยปกติแล้วไม่สามารถทำได้) จะได้
\begin{eqnobox}
    \pdv{u}{t}=-\gd\cdot\vb{S}
\end{eqnobox}
คล้ายกับเป็น ``สมการความต่อเนื่อง'' ของพลังงาน

โดยสมการนี้ก็คือสมการอนุรักษ์พลังงานของแม่เหล็กไฟฟ้านั่นเอง เพราะเมื่ออินทิเกรตบน $\vol$ ที่ใหญ่มาก ๆ (หรือก็คืออินทิเกรตทั่วทุกพื้นที่) จะได้ว่าอินทิกรัล\emph{ฟลักซ์พลังงาน} (อินทิกรัลที่มี Poynting vector)
\begin{eqnobox}
    \odv{W}{t} = -\odv{U_\txt{em}}{t}
\end{eqnobox}
กล่าวคือ พลังงานที่หายไปจากสนาม จะถูกถ่ายทอดมาเป็นงานที่กระทำกับประจุ

\section{โมเมนตัม}

\subsection{เทนเซอร์ความเค้นของ Maxwell}

ประจุที่เคลื่อนที่จะทำให้เกิดสนามไฟฟ้าพุ่งออกจากประจุและสนามแม่เหล็กเป็นไปตามกฎมือขวา (จะพิสูจน์ในบทการแผ่รังสี) เราจะเห็นได้ว่าแรงแม่เหล็กบนสองประจุที่เคลื่อนที่เข้าหากันในแนวตั้งฉากจะไม่เป็นไปตามกฎข้อที่สามของนิวตัน ดังนั้นจริง ๆ แล้วกฎข้อที่สามไม่เป็นจริงสำหรับแม่เหล็กไฟฟ้า จึงทำให้โมเมนตัมแบบดั้งเดิมไม่อนุรักษ์เช่นกัน เราจึงอาจจะหาพจน์โมเมนตัมบางอย่างที่เกิดจากสนามเพื่อทำให้โมเมนตัมอนุรักษ์

พิจารณาแรงทั้งหมดที่สนามกระทำใน $\vol$:
\begin{equation}
    \vb{F} = \int_\vol \ab(\vb{E} + \vb{v}\times\vb{B})\rho\odif{\tau} = \int_\vol\ab(\rho\vb{E} + \vb{J}\times\vb{B})\odif{\tau}\tag{$\star$}\label{forcetensor}
\end{equation}
พิจารณาการเขียนแรงต่อปริมาตร $\vb{f} = \rho\vb{E} + \vb{J}\times\vb{B}$ ในรูปที่ติดแค่ $\vb{E}$ และ $\vb{B}$ จะได้
\begin{align*}
    \vb{f} &= \eps_0\ab(\gd\cdot\vb{E})\vb{E} + \ab(\frac{1}{\mu_0}\ab(\gd\times\vb{B}) - \eps_0\pdv{\vb{E}}{t})\times\vb{B}\\
    &= \eps_0\ab(\gd\cdot\vb{E})\vb{E} - \frac{1}{\mu_0}\vb{B}\times\ab(\gd\times\vb{B}) - \eps_0\pdv{\vb{E}}{t}\times\vb{B}\tag{$\star$1}\label{tensorproof}
\end{align*}
เนื่องจาก
\begin{align*}
    \pdv{\vb{E}}{t}\times\vb{B} &= \pdv{}{t}\ab(\vb{E}\times\vb{B}) - \vb{E}\times\pdv{\vb{B}}{t}\\
    &= \pdv{}{t}\ab(\vb{E}\times\vb{B}) + \vb{E}\times\ab(\gd\times\vb{E})
\end{align*}
แทนใน (\ref{tensorproof}) จะได้ (บวกเข้าด้วย $1/\mu_0\ab(\gd\cdot\vb{B})\vb{B}$)
\begin{equation}
    \vb{f} = \eps_0\ab\Big(\ab(\gd\cdot\vb{E})\vb{E} - \vb{E}\times\ab(\gd\times\vb{E})) + \frac{1}{\mu_0}\ab\Big(\ab(\gd\cdot\vb{B})\vb{B} - \vb{B}\times\ab(\gd\times\vb{B})) - \eps_0\pdv{}{t}\ab(\vb{E}\times\vb{B})\tag{$\star$2}\label{tensorproof2}
\end{equation}
ต่อมา เนื่องจาก
\begin{align*}
    \gd\ab(\vb{E}\cdot\vb{E}) &= \ab(\vb{E}\cdot\gd)\vb{E} + \ab(\vb{E}\cdot\gd)\vb{E} + \vb{E}\times\ab(\gd\times\vb{E}) + \vb{E}\times\ab(\gd\times\vb{E})\\
    \gd\ab(E^2) &= 2\ab(\vb{E}\cdot\gd)\vb{E} + 2\vb{E}\times\ab(\gd\times\vb{E})\\
    \vb{E}\times\ab(\gd\times\vb{E}) &= \frac{1}{2}\gd\ab(E^2) - \ab(\vb{E}\cdot\gd)\vb{E}
\end{align*}
แทนใน (\ref{tensorproof2}) ก็จะได้
\begin{align*}
    \vb{f} =\ \ &\eps_0\ab\Big(\ab(\gd\cdot\vb{E})\vb{E} + \ab(\vb{E}\cdot\gd)\vb{E}) + \frac{1}{\mu_0}\ab\Big(\ab(\gd\cdot\vb{B})\vb{B} + \ab(\vb{B}\cdot\gd)\vb{B})\\ 
    &-\frac{1}{2}\gd\ab(\eps_0E^2 + \frac{1}{\mu_0}B^2)-\eps_0\pdv{}{t}\ab(\vb{E}\times\vb{B})\tag{$\star$3}\label{tensorproof3}
\end{align*}
สามารถเขียนได้อยู่ในรูปที่ง่ายกว่าโดยเราจะนิยาม\emph{เทนเซอร์ความเค้นของ Maxwell} (\emph{Maxwell's stress tensor}) ดังนี้
\begin{defbox}{เทนเซอร์ความเค้นของ Maxwell}
    \begin{equation*}
        T_{ij} \equiv \eps_0\ab(E_iE_j - \frac{1}{2}\delta_{ij}E^2) + \frac{1}{\mu_0}\ab(B_iB_j - \frac{1}{2}\delta_{ij}B^2)
    \end{equation*}
    เมื่อ $\delta_{ij}$ คือ \emph{Kronecker delta} ซึ่งเท่ากับ $1$ เมื่อ $i=j$ มิฉะนั้นเป็น $0$
\end{defbox}
โดยสังเกตว่า
\begin{align*}
    \ab(\gd\cdot\tsr{T})_j &= \sum_{i=x,y,z}\pdv{}{i}T_{ij}\\
    &= \eps_0\ab(\ab(\gd\cdot\vb{E})E_j + \ab(\vb{E}\cdot\gd)E_j-\frac{1}{2}\nabla_jE^2) + \frac{1}{\mu_0}\ab(\ab(\gd\cdot\vb{B})B_j + \ab(\vb{B}\cdot\gd)B_j-\frac{1}{2}\nabla_jB^2)
\end{align*}
(เมื่อ $(\vb{a}\cdot \tsr{T})_j = a_iT_{ij}$ คือ column-wise dot product) จาก (\ref{tensorproof3}) จึงได้ว่า
\begin{eqbox}{แรงต่อปริมาตร}
    \vb{f} = \gd\cdot\tsr{T} - \mu_0\eps_0\pdv{\vb{S}}{t}
\end{eqbox}
ดังนั้นแทนกลับใน (\ref{forcetensor}) จะได้
\begin{eqnobox}
    \vb{F} = \oint_{\del\vol}\tsr{T}\cdot\odif{\vb{a}} - \mu_0\eps_0\odv{}{t}\int_\vol\vb{S}\odif{\tau}\label{forcestress}
\end{eqnobox}
โดยถ้าระบบเป็นระบบสนามสถิตก็จะได้ว่า
\begin{eqnobox}
    \vb{F} = \oint_{\del\vol}\tsr{T}\cdot\odif{\vb{a}}
\end{eqnobox}

\subsection{กฎอนุรักษ์โมเมนตัม}

ถ้าเรานิยาม\emph{โมเมนตัมจากสนามแม่เหล็กไฟฟ้า}ว่า
\begin{defbox}{โมเมนตัมจากสนามแม่เหล็กไฟฟ้า}
    \begin{equation*}
        \vb{p}_\txt{em} \equiv \mu_0\eps_0\int \vb{S}\odif{\tau}
    \end{equation*}
\end{defbox}
โดยมี\emph{ความหนาแน่นโมเมนตัม} ($\vb{g}$)
\begin{defbox}{ความหนาแน่นโมเมนตัม}
    \begin{equation*}
        \vb{g} \equiv \mu_0\eps_0\vb{S} = \eps_0\ab(\vb{E}\times\vb{B})
    \end{equation*}
\end{defbox}
จาก (\ref{forcestress}) ก็จะได้
\begin{ieqbox}{แรงลัพธ์ในรูปเทนเซอร์ความเค้น}
    \vb{F} = -\odv{}{t}\int_\vol\vb{g}\odif{\tau} + \oint_{\del\vol}\tsr{T}\cdot\odif{\vb{a}}
\end{ieqbox}
และจะได้เป็นกฎอนุรักษ์โมเมนตัม:
\begin{ieqbox}{กฎอนุรักษ์โมเมนตัม}
    \odv{\vb{p}_\txt{mech}}{t} = -\odv{\vb{p}_\txt{em}}{t} + \oint\tsr{T}\cdot\odif{\vb{a}}
\end{ieqbox}
และถ้าไม่เกิดการเปลี่ยนแปลงโมเมนตัมเชิงกล จะได้ ``สมการความต่อเนื่อง'' ของโมเมนตัม:
\begin{eqnobox}
    \pdv{\vb{g}}{t} = \gd\cdot\tsr{T}
\end{eqnobox}
สุดท้าย เช่นเดียวกับกฎอนุรักษ์พลังงาน พจน์\emph{ฟลักซ์โมเมนตัม} (คือพจน์ที่มี stress tensor, โดยในกรณีนี้ flux density จะเป็น $-\tsr{T}$ เพราะในสมการเป็นบวก) จะเข้าใกล้ศูนย์ถ้าอินทิเกรตบนพื้นที่ที่ใหญ่มาก ๆ (อินทิเกรตทั่วทุกพื้นที่) ทำให้เกิดการอนุรักษ์โมเมนตัม
\begin{eqnobox}
    \odv{\vb{p}_\txt{mech}}{t} = -\odv{\vb{p}_\txt{em}}{t}
\end{eqnobox}

\subsection{โมเมนตัมเชิงมุม}

เราสามารถนิยามโมเมนตัมเชิงมุมของสนามแม่เหล็กไฟฟ้ารอบจุด ๆ หนึ่งได้เช่นกัน
\begin{defbox}{ความหนาแน่นโมเมนตัมเชิงมุม}
    \begin{equation*}
        \vb{l} \equiv \vb{r}\times\vb{g} = \eps_0\ab\big(\vb{r}\times\ab(\vb{E}\times\vb{B}))
    \end{equation*}
\end{defbox}
ก็จะได้
\begin{defbox}{โมเมนตัมเชิงมุมจากสนามแม่เหล็กไฟฟ้า}
    \begin{equation*}
        \vb{L}_\txt{em} \equiv \int\vb{l}\odif{\tau}
    \end{equation*}
\end{defbox}
โดยโมเมนตัมเชิงมุมนี้ก็อนุรักษ์กับ counterpart ของมันเช่นเดียวกับพลังงานและโมเมนตัมเชิงเส้น โดยจากทอร์ก:
\begin{align*}
    \vbs{\uptau} &= \int_\vol \vb{r} \times \vb{f}\odif{\tau}\\
    &= \int_\vol \vb{r}\times\ab(\gd\cdot\tsr{T})\odif{\tau} - \odv{}{t}\int_\vol \vb{r}\times\vb{g}\odif{\tau}
\end{align*}
เนื่องจาก
\[
\int_\vol\vb{r}\times\ab(\gd\times\tsr{T})\odif{\tau} = -\int_\vol\cancel{\ab(\gd\times\vb{r})}\cdot\tsr{T}\odif{\tau} + \int_\vol\gd\cdot\ab(\vb{r}\times\tsr{T})\cdot\odif{\tau} = \oint_{\del\vol}\ab(\vb{r}\times\tsr{T})\cdot\odif{\vb{a}}
\]
(เมื่อ $\vb{r}\times\tsr{T}$ เป็น column-wise cross product) ก็จะได้ว่า
\begin{ieqbox}{ทอร์กในรูปเทนเซอร์ความเค้น}
    \vbs{\uptau} = -\odv{}{t}\int_\vol\vb{l}\odif{\tau} + \oint_{\del\vol}\ab(\vb{r}\times\tsr{T})\cdot\odif{\vb{a}}
\end{ieqbox}
หรือ
\begin{ieqbox}{กฎอนุรักษ์โมเมนตัมเชิงมุม}
    \odv{\vb{L}_\txt{mech}}{t} = -\odv{\vb{L}_\txt{em}}{t} + \oint\ab(\vb{r}\times\tsr{T})\cdot\odif{\vb{a}}
\end{ieqbox}
ถ้าไม่เกิดการเปลี่ยนแปลงโมเมนตัมเชิงมุม (เชิงกล) จะได้ ``สมการความต่อเนื่อง'' ของโมเมนตัมเชิงมุม:
\begin{eqnobox}
    \pdv{\vb{l}}{t} = \gd\cdot\ab(\vb{r}\times\tsr{T})
\end{eqnobox}
และเนื่องจากพจน์\emph{ฟลักซ์โมเมนตัมเชิงมุม} ($-\oint(\vb{r}\times\tsr{T})\cdot\odif{\vb{a}}$) จะเข้าใกล้ศูนย์ถ้าอินทิเกรตบนทุกพื้นที่ ทำให้เกิดการอนุรักษ์โมเมนตัมมุม
\begin{eqnobox}
    \odv{\vb{L}_\txt{mech}}{t} = -\odv{\vb{L}_\txt{em}}{t}
\end{eqnobox}