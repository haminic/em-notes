\chapter{วงจรไฟฟ้ากระแส}
\section{การวิเคราะห์วงจร}
\subsection{การต่อตัวเก็บประจุ, ตัวต้านทาน, และตัวเหนี่ยวนำอย่างง่าย}

พิจารณาการต่อตัวเก็บประจุ $C_1$ และ $C_2$ แบบ\emph{อนุกรม} จะได้ว่า $Q$ บนตัวเก็บประจุ $C_1$ จะเท่ากับ $Q$ บนตัวเก็บประจุ $C_2$ ดังนั้น
\[ 
V_\txt{total}=V_1+V_2=\frac{Q}{C_1}+\frac{Q}{C_2}
\]
จึงได้ความจุไฟฟ้ารวมเท่ากับ
\[ 
\frac{1}{C_\txt{total}}=\frac{1}{C_1}+\frac{1}{C_2}
\]
โดยเราสามารถทำแบบนี้ไปได้เรื่อย ๆ ด้วยตัวเก็บประจุกี่ตัวก็ได้ ดังนั้น
\begin{eqbox}{การต่อตัวเก็บประจุแบบอนุกรม}
    \frac{1}{C_\txt{total}}=\frac{1}{C_1}+\frac{1}{C_2}+\dots+\frac{1}{C_n}
\end{eqbox}
และพิจารณาการต่อตัวเก็บประจุ $C_1$ และ $C_2$ แบบ\emph{ขนาน} จะได้ว่าความต่างศักย์ของตัวเก็บประจุทั้งสองจะต้องเท่ากัน (เพราะเป็นเนื้อตัวนำเดียวกัน) ดังนั้น
\[ 
Q_\txt{total}=Q_1+Q_2=C_1V+C_2V
\]
จึงได้ความจุไฟฟ้ารวมเท่ากับ
\[ 
C_\txt{total}=C_1+C_2
\]
โดยเช่นเดียวกับการต่อแบบอนุกรม เราสามารถทำแบบนี้ไปได้เรื่อย ๆ ด้วยตัวเก็บประจุกี่ตัวก็ได้ ดังนั้น
\newpage
\begin{eqbox}{การต่อตัวเก็บประจุแบบขนาน}
    C_\txt{total}=C_1+C_2+\dots+C_n
\end{eqbox}

ต่อมาเช่นเดียวกับตัวเก็บประจุ พิจารณาการต่อตัวต้านทาน $R_1$ และ $R_2$ แบบอนุกรม จะได้ว่ากระแสไฟฟ้าที่ไหลผ่านตัวต้านทานทั้งสองจะต้องเท่ากัน ดังนั้น
\[ 
V_\txt{total}=V_1+V_2=IR_1+IR_2=IR_1+IR_2
\]
เราสามารถทำแบบนี้ไปได้เรื่อย ๆ ด้วยตัวต้านทานกี่ตัวก็ได้ ดังนั้น
\begin{eqbox}{การต่อตัวต้านทานแบบอนุกรม}
    R_\txt{total}=R_1+R_2+\dots+R_n
\end{eqbox}
และพิจารณาการต่อตัวต้านทาน $R_1$ และ $R_2$ แบบขนาน จะได้ว่าความต่างศักย์ของตัวต้านทานทั้งสองจะต้องเท่ากัน ดังนั้น
\[ 
I_\txt{total}=I_1+I_2=\frac{V}{R_1}+\frac{V}{R_2}
\]
เราสามารถทำแบบนี้ไปได้เรื่อย ๆ ด้วยตัวต้านทานกี่ตัวก็ได้ ดังนั้น
\begin{eqbox}{การต่อตัวต้านทานแบบขนาน}
    \frac{1}{R_\txt{total}}=\frac{1}{R_1}+\frac{1}{R_2}+\dots+\frac{1}{R_n}
\end{eqbox}

สุดท้าย พิจารณาการต่อตัวเหนี่ยวนำ $L_1$ และ $L_2$ แบบอนุกรม จะได้ว่ากระแสที่ไหลผ่านตัวเหนี่ยวนำทั้งสองจะต้องเท่ากัน ดังนั้น
\[ 
-\emf_\txt{total}=-\emf_1-\emf_2=L_1\odv{I}{t}+L_2\odv{I}{t}=L_1\odv{I}{t}+L_2\odv{I}{t}
\]
ก็จะได้
\begin{eqbox}{การต่อตัวเหนี่ยวนำแบบอนุกรม}
    L_\txt{total}=L_1+L_2+\dots+L_n
\end{eqbox}
และพิจารณาการต่อตัวเหนี่ยวนำ $L_1$ และ $L_2$ แบบขนาน จะได้ความต่างศักย์ (ในที่นี้เท่ากับ emf) บนตัวเหนี่ยวนำทั้งสองเท่ากัน ดังนั้น
\[
\odv{I_\txt{total}}{t} = \odv{I_1}{t} + \odv{I_2}{t} = - \frac{\emf}{L_1} - \frac{\emf}{L_2}
\]
ก็จะได้
\begin{eqbox}{การต่อตัวเหนี่ยวนำแบบขนาน}
    \frac{1}{L_\txt{total}}=\frac{1}{L_1}+\frac{1}{L_2}+\dots+\frac{1}{L_n}
\end{eqbox}

โดยเรามี ``กฎของ Ohm'' สำหรับแต่ละ \emph{passive component} (อุปกรณ์ที่ไม่สร้างพลังงาน) ดังนี้:
\begin{lawbox}{ความสัมพันธ์ของ $V$ และ $I$ สำหรับ Passive Component}
    สำหรับดัวเก็บประจุ:
    \begin{equation}
        I = C\odv{V}{t} \label{begohm}
    \end{equation}
    สำหรับตัวต้านทาน:
    \begin{equation}
        V = IR
    \end{equation}
    และสำหรับตัวเหนี่ยวนำ:
    \begin{equation}
        V = I\odv{I}{t} \label{endohm}
    \end{equation}
\end{lawbox}

\subsection{กฎของ Kirchhoff}

พิจารณาวงจร ณ จุด ๆ หนึ่ง ถ้าที่จุดนั้นไม่มีประจุสะสมอยู่เลยโดย \ref{contj} จะได้ว่า
\begin{ieqbox}{Kirchhoff's Current Law (KCL)}
    \sum I_\txt{in} = \sum I_\txt{out}\label{kcl}
\end{ieqbox}

โดยกฎนี้ใช้ในการวิเคราะห์วงจรแบบ\emph{โนด} (\emph{node analysis}) โดยเริ่มจากการตั้งศักย์ไฟฟ้าบนแต่ละโนดและกระแสที่ไหลเข้าและออกจากแต่ละโนด จากนั้นใช้ (\ref{kcl}) และ (\ref{begohm}) ถึง (\ref{endohm}) ในการเขียนทุกตัวแปรให้อยู่ในรูป $V$

ต่อมาพิจารณาวงจรที่ไม่มีการเปลี่ยนแปลงสนามแม่เหล็ก จะได้ว่า $\vb{E}$ เป็นสนามอนุรักษ์ ดังนั้น
\begin{ieqbox}{Kirchhoff's Voltage Law (KVL)}
    \sum_\txt{loop} V = 0\label{kvl}
\end{ieqbox}
(เมื่อ $V$ คือความต่างศักย์)

กฎนี้ใช้ในการวิเคราะห์วงจรแบบ\emph{ลูป} (\emph{mesh analysis}) โดยเริ่มจากกำหนดกระแสที่วนอยู่ในแต่ละลูปที่กำหนดขึ้น จากนั้นใช้ (\ref{kvl}) และ (\ref{begohm}) ถึง (\ref{endohm}) ตั้งสมการตามจำนวนลูปที่กำหนดไว้เพื่อแก้หา $I$ ในแต่ละลูป

\section{วงจรไฟฟ้ากระแสตรง}

\subsection{อันดับของวงจร}

\begin{defbox}{อันดับของวงจร}
    \emph{อันดับของวงจร}คืออันดับของสมการเชิงอนุพันธ์ที่อธิบายวงจร เช่นวงจรไฟฟ้ากระแสตรงที่มีแค่แบตเตอรี่และตัวต้านทานไม่มีอนุพันธ์อะไรเลย จึงเป็นวงจรอันดับศูนย์
\end{defbox}

โดยวงจรอันดับหนึ่งได้แก่ วงจรที่มีตัวต้านทานและตัวเก็บประจุ (วงจร RC) และวงจรที่มีตัวต้านทานและตัวเหนี่ยวนำ (วงจร RL) และวงจรอันดับสองได้แก่วงจรที่มี passive component ทั้งสาม (วงจร RLC)

\subsection{วงจร RC}

วงจร RC เป็นวงจรอันดับหนึ่ง โดยจะยกตัวอย่างโจทย์การปล่อยประจุจากตัวเก็บประจุ:

\begin{corbox}{ตัวอย่าง}
    จงหา I(t) ของวงจรที่มีการต่อตัวเก็บประจุ $C$ และตัวต้านทาน $R$ แบบอนุกรม โดยที่ $C$ มีประจุเริ่มต้น $Q_0$
\end{corbox}

\begin{soln}
    ให้
\end{soln}

