\chapter{วงจรไฟฟ้า}
\section{การวิเคราะห์วงจร}

\subsection{กฎของ Kirchhoff}

เรามี ``กฎของ Ohm'' สำหรับแต่ละ \emph{passive component} (อุปกรณ์ที่ไม่สร้างพลังงาน) ดังนี้
\begin{lawbox}{ความสัมพันธ์ของ $V$ และ $I$ สำหรับ Passive Component}
    สำหรับดัวเก็บประจุ:
    \begin{equation}
        i = C\odv{v}{t} \label{begohm}
    \end{equation}
    สำหรับตัวต้านทาน:
    \begin{equation}
        v = iR
    \end{equation}
    และสำหรับตัวเหนี่ยวนำ:
    \begin{equation}
        v = L\odv{i}{t} \label{endohm}
    \end{equation}
\end{lawbox}
(ในบทนี้เราจะใช้ตัวอักษร $v$ และ $i$ ที่เป็นตัวพิมพ์เล็กเพื่อแทนความต่างศักย์และกระแสที่อาจขึ้นกับเวลา) ซึ่งสามารถนำมาใช้ในการวิเคราะห์วงจรได้ด้วยกฎของ Kirchhoff:

พิจารณาวงจร ณ จุด ๆ หนึ่ง ถ้าที่จุดนั้นไม่มีประจุสะสมอยู่เลยโดย \ref{contj} จะได้ว่า
\begin{ieqbox}{Kirchhoff's Current Law (KCL)}
    \sum i_\txt{in} = \sum i_\txt{out}\label{kcl}
\end{ieqbox}

โดยกฎนี้ใช้ในการวิเคราะห์วงจรแบบ\emph{โนด} (\emph{nodal analysis}) โดยเริ่มจากการตั้งศักย์ไฟฟ้าบนแต่ละโนดและกระแสที่ไหลเข้าและออกจากแต่ละโนด จากนั้นใช้ (\ref{kcl}) และ (\ref{begohm}) ถึง (\ref{endohm}) ในการเขียนทุกตัวแปรให้อยู่ในรูป $V$

ต่อมาพิจารณาวงจรที่ไม่มีการเปลี่ยนแปลงสนามแม่เหล็ก จะได้ว่า $\vb{E}$ เป็นสนามอนุรักษ์ ดังนั้น
\begin{ieqbox}{Kirchhoff's Voltage Law (KVL)}
    \sum_\txt{loop} v = 0\label{kvl}
\end{ieqbox}
หมายเหตุ: \emph{สังเกตว่าจาก KCL (\ref{kcl}) และสมบัติเชิงเส้นของ (\ref{begohm}) ถึง (\ref{endohm}) จริง ๆ แล้วความต่างศักย์นี้ไม่จำเป็นจะต้องคำนวณจากกระแสรวม แต่ขอแค่เป็นกระแสสมมติที่ครบวงปิดก็พอ}

กฎนี้ใช้ในการวิเคราะห์วงจรแบบ\emph{ลูป} (\emph{mesh analysis}) โดยเริ่มจากกำหนดกระแสที่วนอยู่ในแต่ละลูปที่กำหนดขึ้น จากนั้นใช้ (\ref{kvl}) และ (\ref{begohm}) ถึง (\ref{endohm}) ตั้งสมการตามจำนวนลูปที่กำหนดไว้เพื่อแก้หา $I$ ในแต่ละลูป

\subsection{การต่อตัวเก็บประจุ, ตัวต้านทาน, และตัวเหนี่ยวนำอย่างง่าย}

พิจารณาการต่อตัวเก็บประจุ $C_1$ และ $C_2$ แบบ\emph{อนุกรม} จะได้ว่า $q$ บนตัวเก็บประจุ $C_1$ จะเท่ากับ $q$ บนตัวเก็บประจุ $C_2$ ดังนั้น
\[ 
v_\txt{total}=v_1+v_2=\frac{q}{C_1}+\frac{q}{C_2}
\]
จึงได้ความจุไฟฟ้ารวมเท่ากับ
\[ 
\frac{1}{C_\txt{total}}=\frac{1}{C_1}+\frac{1}{C_2}
\]
โดยเราสามารถทำแบบนี้ไปได้เรื่อย ๆ ด้วยตัวเก็บประจุกี่ตัวก็ได้ ดังนั้น
\begin{eqbox}{การต่อตัวเก็บประจุแบบอนุกรม}
    \frac{1}{C_\txt{total}}=\frac{1}{C_1}+\frac{1}{C_2}+\dots+\frac{1}{C_n}
\end{eqbox}
และพิจารณาการต่อตัวเก็บประจุ $C_1$ และ $C_2$ แบบ\emph{ขนาน} จะได้ว่าความต่างศักย์ของตัวเก็บประจุทั้งสองจะต้องเท่ากัน (เพราะเป็นเนื้อตัวนำเดียวกัน) ดังนั้น
\[ 
Q_\txt{total}=Q_1+Q_2=C_1v+C_2v
\]
จึงได้ความจุไฟฟ้ารวมเท่ากับ
\[ 
C_\txt{total}=C_1+C_2
\]
โดยเช่นเดียวกับการต่อแบบอนุกรม เราสามารถทำแบบนี้ไปได้เรื่อย ๆ ด้วยตัวเก็บประจุกี่ตัวก็ได้ ดังนั้น
\newpage
\begin{eqbox}{การต่อตัวเก็บประจุแบบขนาน}
    C_\txt{total}=C_1+C_2+\dots+C_n
\end{eqbox}

ต่อมาเช่นเดียวกับตัวเก็บประจุ พิจารณาการต่อตัวต้านทาน $R_1$ และ $R_2$ แบบอนุกรม จะได้ว่ากระแสไฟฟ้าที่ไหลผ่านตัวต้านทานทั้งสองจะต้องเท่ากัน ดังนั้น
\[ 
v_\txt{total}=v_1+v_2=iR_1+iR_2=iR_1+iR_2
\]
เราสามารถทำแบบนี้ไปได้เรื่อย ๆ ด้วยตัวต้านทานกี่ตัวก็ได้ ดังนั้น
\begin{eqbox}{การต่อตัวต้านทานแบบอนุกรม}
    R_\txt{total}=R_1+R_2+\dots+R_n
\end{eqbox}
และพิจารณาการต่อตัวต้านทาน $R_1$ และ $R_2$ แบบขนาน จะได้ว่าความต่างศักย์ของตัวต้านทานทั้งสองจะต้องเท่ากัน ดังนั้น
\[ 
i_\txt{total}=i_1+i_2=\frac{v}{R_1}+\frac{v}{R_2}
\]
เราสามารถทำแบบนี้ไปได้เรื่อย ๆ ด้วยตัวต้านทานกี่ตัวก็ได้ ดังนั้น
\begin{eqbox}{การต่อตัวต้านทานแบบขนาน}
    \frac{1}{R_\txt{total}}=\frac{1}{R_1}+\frac{1}{R_2}+\dots+\frac{1}{R_n}
\end{eqbox}

สุดท้าย พิจารณาการต่อตัวเหนี่ยวนำ $L_1$ และ $L_2$ แบบอนุกรม จะได้ว่ากระแสที่ไหลผ่านตัวเหนี่ยวนำทั้งสองจะต้องเท่ากัน ดังนั้น
\[ 
v_\txt{total} = v_1 + v_2=L_1\odv{i}{t}+L_2\odv{i}{t}=L_1\odv{i}{t}+L_2\odv{i}{t}
\]
ก็จะได้
\begin{eqbox}{การต่อตัวเหนี่ยวนำแบบอนุกรม}
    L_\txt{total}=L_1+L_2+\dots+L_n
\end{eqbox}
และพิจารณาการต่อตัวเหนี่ยวนำ $L_1$ และ $L_2$ แบบขนาน จะได้ความต่างศักย์บนตัวเหนี่ยวนำทั้งสองเท่ากัน ดังนั้น
\[
\odv{i_\txt{total}}{t} = \odv{i_1}{t} + \odv{i_2}{t} = \frac{v}{L_1} + \frac{v}{L_2}
\]
ก็จะได้
\begin{eqbox}{การต่อตัวเหนี่ยวนำแบบขนาน}
    \frac{1}{L_\txt{total}}=\frac{1}{L_1}+\frac{1}{L_2}+\dots+\frac{1}{L_n}
\end{eqbox}

\section{วงจรไฟฟ้ากระแสตรง}

\subsection{อันดับของวงจร}

\begin{defbox}{อันดับของวงจร}
    \emph{อันดับของวงจร}คืออันดับของสมการเชิงอนุพันธ์ที่อธิบายวงจร เช่นวงจรไฟฟ้ากระแสตรงที่มีแค่แบตเตอรี่และตัวต้านทานไม่มีอนุพันธ์อะไรเลย จึงเป็นวงจรอันดับศูนย์
\end{defbox}

โดยวงจรอันดับหนึ่งได้แก่ วงจรที่มีตัวต้านทานและตัวเก็บประจุ (วงจร RC) และวงจรที่มีตัวต้านทานและตัวเหนี่ยวนำ (วงจร RL) และวงจรอันดับสองได้แก่วงจรที่มี passive component ทั้งสาม (วงจร RLC)

\subsection{วงจร RC}

วงจร RC เป็นวงจรอันดับหนึ่ง โดยจะยกตัวอย่างโจทย์การปล่อยประจุ (\emph{discharge}) จากตัวเก็บประจุ:

\begin{corbox}{ตัวอย่าง}
    จงหา $v(t)$ คร่อมตัวเก็บประจุของวงจรที่มีการต่อตัวเก็บประจุ $C$ และตัวต้านทาน $R$ แบบอนุกรม โดยที่ $C$ มีประจุเริ่มต้น $Q_0$
\end{corbox}

\begin{soln}
    ให้ $i_R$ และ $i_C$ คือกระแสที่ไหลออกจากจุด ๆ หนึ่งที่อยู่ฝั่งบวกของตัวเก็บประจุ จากนั้นใช้ KCL จะได้
    \begin{align*}
        i_R + i_C &= 0\\
        \frac{v}{R} + C\odv{v}{t'} &= 0\\
        -\frac{1}{RC}\odif{t'} &= \frac{1}{v} \odif{v}\\
        -\int_0^t \frac{1}{RC}\odif{t'} &= \int_{V_0}^{v(t)} \frac{1}{v} \odif{v}\\
        -\frac{1}{RC}t &= \log\ab(\frac{v(t)}{V_0})
    \end{align*}
    เนื่องจาก $V_0 = Q_0 / C$ ก็จะได้
    \[
    v(t) = \frac{Q_0}{C}\exp\ab(-\frac{1}{RC}t)
    \]
    โยเราจะเรียก $\tau \equiv RC$ ว่า\emph{ค่าคงที่เวลา} (\emph{time constant})
\end{soln}

พิจารณาวงจรที่มี $C$ ที่ \emph{steady state} (เมื่อจงจรเป็น steady current) ก็จะได้ว่า
\[
i_C = C\cancel{\odv{v}{t}} = 0 
\]
ดังนั้นเมื่อ $t\to\infty$ จะสามารถมองได้ว่า $C$ เปรียบเสมือนสายไฟขาด

\subsection{วงจร RL}

วงจร RL เป็นวงจรอันดับหนึ่ง โดยจะยกตัวอย่างโจทย์การต่อแบตเตอรี่กับวงจรที่มี $L$:

\begin{corbox}{ตัวอย่าง}
    จงหา $i(t)$ และค่าคงที่เวลา $\tau$ ของการต่อแบตเตอรี่ที่มีแรงเคลื่อนไฟฟ้า $\emf$ ในวงจรที่มีการต่อตัวต้านทาน $R$ และตัวเหนี่ยวนำ $L$ แบบอนุกรม โดยที่ ณ เวลา $t = 0$ ไม่มีกระแสไหลอยู่เลย
\end{corbox}

\begin{soln}
    วนลูปที่มีกระแส $i(t)$ รอบวงจร จากนั้นใช้ KVL จะได้
    \begin{align*}
        v_R + v_L - \emf &= 0\\
        iR + L\odv{i}{t'} &= \emf\\
        -\frac{1}{L} \odif{t'} &= \frac{1}{iR-\emf} \odif{i}\\
        -\int_0^t \frac{1}{L} \odif{t'} &= \int_0^{i(t)} \frac{1}{iR-\emf}\odif{i}\\
        -\frac{R}{L}t &= \log\ab(\frac{\emf - Ri(t)}{\emf})
    \end{align*}
    ดังนั้นก็จะได้
    \[
    i(t) = \frac{\emf}{R}\ab(1-\exp\ab(-\frac{R}{L}t))
    \]
    และค่าคงที่เวลา $\tau = L / R$
\end{soln}

พิจารณาวงจรที่มี $L$ ที่ steady state ก็จะได้ว่า
\[
v = L\cancel{\odv{i}{t}} = 0 
\]
ดังนั้นเมื่อ $t\to\infty$ จะสามารถมองได้ว่า $L$ เปรียบเสมือนสายไฟเปล่า

\subsection{วงจร RLC แบบอนุกรม}

พิจารณากระแก้สมการของวงจรที่เป็น RLC ที่ต่อแบบอนุกรมโดยไม่มีแบตเตอรี่ โดย KVL จะได้ (ให้กระแสไหลออกจากฝั่งลบของตัวเก็บประจุ)
\begin{align*}
    v_R + v_L + v_C &= 0\\
    \odv{}{t}\ab(v_R + v_L + v_C) &= 0\\
    L\odv[2]{i}{t} + R\odv{i}{t} + \frac{1}{C}i &= 0
\end{align*}

เราสามารถแก้สมการนี้ได้โดยการใช้ราก $s_{1,2}$ ของ characteristic equation:
\[
x^2 + (R/L)x + 1/LC =0
\]
ก็จะได้คำตอบคือ
\begin{eqbox}{คำตอบของ Characteristic Equation ของ RLC แบบอนุกรม}
    s_{1,2} = -\frac{R}{2L}\pm\sqrt{\ab(\frac{R}{2L})^2 - \frac{1}{LC}}
\end{eqbox}

โดยเราจะนิยามราก $s_{1,2}$ ของสมการว่าเป็น\emph{ความถี่ธรรมชาติ}ของวงจร และก็จะนิยาม \emph{damping factor} ($\alpha$) และ\emph{ความถี่ resonant} ($\omega_0$) (\emph{undamped natural frequency}) ดังนี้:
\begin{defbox}{ Damping Factor และความถี่ Resonant}
    \begin{equation}
    s_{1,2} \equiv -\alpha \pm \sqrt{\alpha^2 - \omega_0^2}
    \end{equation}
\end{defbox}
หมายเหตุ: \emph{โดยเราจะใช้หน่วย $\txt{Np}/\txt{s}$ สำหรับ $\alpha$ แต่จริง ๆ แล้วหน่วย $\txt{Np}$ (neper) นี้เป็นน่วยที่ไม่มีมิติเหมือนกับ $\txt{rad}$}

ในที่นี้ก็จะได้
\begin{eqbox}{Damping Factor และความถี่ Resonant ของ RLC แบบอนุกรม}
    \alpha = \frac{R}{2L}\qq{และ}\omega_0 = \frac{1}{\sqrt{LC}}
\end{eqbox}

เมื่อแก้สมการเชิงอนุพันธ์ ถ้าราก $s_{1,2}$ เป็นจำนวนจริง ($\alpha > \omega_0$) จะได้ว่า
\begin{eqbox}{วงจร RLC อนุกรมแบบ Underdamped}
    i(t) = A_1\exp(s_1 t) + A_2\exp(s_2 t)
\end{eqbox}
โดยเราจะเรียกว่าเป็นวงจร \emph{underdamped}

แต่ถ้าราก $s_{1,2}$ เป็นรากซ้ำ ($\alpha = \omega_0$) โดยให้เป็น $s$ จะได้ว่า
\begin{eqbox}{วงจร RLC อนุกรมแบบ Critically Damped}
    i(t) = (A_2 + A_1t)\exp(st)
\end{eqbox}
โดยเราจะเรียกว่าเป็นวงจร \emph{critically damped}

และสุดท้าย ถ้าราก $s_{1,2}$ ไม่เป็นจำนวนจริง ($\alpha < \omega_0$) เราจะนิยาม\emph{ความถี่ damped} (\emph{damped natural frequency}):
\begin{defbox}{ความถี่ Damped}
    \begin{equation}
        \omega_d \equiv \sqrt{\omega_0^2 - \alpha^2}
    \end{equation}
\end{defbox}
และก็จะได้คำตอบของสมการว่า
\begin{eqbox}{วงจร RLC อนุกรมแบบ Overdamped}
    i(t) = \exp(-\alpha t) \left( A_1 \cos(\omega_d t) + A_2 \sin(\omega_d t) \right)
\end{eqbox}
โดยเราจะเรียกว่าเป็นวงจร \emph{overdamped}

\subsection{วงจร RLC แบบขนาน}

พิจารณากระแก้สมการของวงจรที่เป็น RLC ที่ต่อแบบขนานโดยไม่มีแบตเตอรี่ โดย KCL จะได้ (ให้กระแสไหลออกจากฝั่งลบของตัวเก็บประจุ)
\begin{align*}
    i_R + i_L + i_C &= 0\\
    \odv{}{t}\ab(i_R + i_L + i_C) &= 0\\
    C\odv[2]{v}{t} + \frac{1}{R}\odv{v}{t} + \frac{1}{L}v &= 0
\end{align*}

ก็จะได้ characteristic equation:
\[
x^2 + (1/RC)x + 1/LC = 0
\]
มีคำตอบ $s_{1,2}$ คือ
\begin{eqbox}{คำตอบของ Characteristic Equation ของ RLC แบบขนาน}
    s_{1,2} = -\frac{1}{2RC}\pm\sqrt{\ab(\frac{1}{2RC})^2 - \frac{1}{LC}}
\end{eqbox}
ก็จะได้ว่า
\begin{eqbox}{Damping Factor และความถี่ Resonant ของ RLC แบบขนาน}
    \alpha = \frac{1}{2RC}\qq{และ}\omega_0 = \frac{1}{\sqrt{LC}}
\end{eqbox}

ต่อมาเมื่อแก้สมการเชิงอนุพันธ์ (เช่นเดียวกับในกรณีต่อแบบอนุกรมแต่ทีนี้หา $v$ แทน $i$) ก็จะได้
\begin{eqbox}{วงจร RLC ขนานแบบ Underdamped}
    v(t) = A_1\exp(s_1 t) + A_2\exp(s_2 t)
\end{eqbox}
ในกรณี underdamped,
\begin{eqbox}{วงจร RLC ขนานแบบ Critically Damped}
    v(t) = (A_2 + A_1t)\exp(st)
\end{eqbox}
ในกรณี critically damped, และ
\begin{eqbox}{วงจร RLC ขนานแบบ Overdamped}
    v(t) = \exp(-\alpha t) \left( A_1 \cos(\omega_d t) + A_2 \sin(\omega_d t) \right)
\end{eqbox}
ในกรณี overdamped

\section{วงจรไฟฟ้ากระแสสลับ}

\subsection{แหล่งกำเนิดไฟฟ้ากระแสสลับ}

วงจรทั้งหมดที่เราเจอมาเรียกว่า\emph{วงจรไฟฟ้ากระแสตรง} (\emph{direct current} หรือ \emph{DC}) ซึ่งเกิดจากแหล่งกำเนิดที่ให้ความต่างศักย์คงที่ (หรือกระแสคงที่) ตลอดเวลา แต่วงจรไฟฟ้าที่เราใช้ในบ้านจะเป็น\emph{วงจรไฟฟ้ากระแสสลับ} (\emph{alternating current} หรือ \emph{AC}) ซึ่งเป็นวงจรที่แหล่งกำเนิดมี emf สลับทิศไปมาเป็นฟังก์ชันคาบ โดยไฟฟ้ากระแสสลับที่เราจะมาดูกันเราจะสมมติมาเป็นไฟฟ้ากระแสสลับรูปไซน์ (sinusoidal) ก็คือ
\begin{eqbox}{ไฟฟ้ากระแสสลับ}
    (i\txt{\ or\ }v)(t) = (I\txt{\ or\ }V)_m\cos \omega t
\end{eqbox}
เมื่อ $I_m$ และ $V_m$ คือ\emph{แอมพลิจูด}ของกราฟไซน์และ $\omega$ เรียกว่า\emph{ความถี่เชิงมุม} โดย argument ที่อยู่ในฟังก์ชันเราจะเรียกว่า\emph{เฟส} ณ ขณะนั้น และเราจะนิยาม\emph{คาบ} ($T$) และ\emph{ความถี่} ($f$) ตามปกติ:
\begin{defbox}{คาบและความถี่}
    \begin{equation}
        T\equiv\frac{2\pi}{\omega}\qq{และ}f\equiv\frac{1}{T} = \frac{\omega}{2\pi}
    \end{equation}
\end{defbox}

โดยยกตัวอย่างแหล่งกำเนิดไฟฟ้ากระแสสลับเช่น ลองพิจารณาสายไฟที่หมุนรอบแกน $x$ และมีสนามแม่เหล็กสม่ำเสมอขนาด $B$ ชี้ในแกน $+y$ ก็จะได้ว่าถ้าเราหมุนลูปสายไฟนี้ด้วยอัตราเร็วเชิงมุม $\omega$ แล้วก็จะได้ว่า
\[
\Phi_B = BA\cos\omega t
\]
ดังนั้น
\[
\emf = -\odv{\Phi_B}{t} = BA\omega \sin \omega t
\]
\begin{eqbox}{แหล่งกำเนิดไฟฟ้ากระแสสลับ}
    \emf = BA\omega\sin\omega t = BA\omega\cos\ab(\omega t + \pi/2)
\end{eqbox}

\subsection{ตัวต้านทาน, ตัวเก็บประจุ, และตัวเหนี่ยวนำ}

ต่อมาเรามาพิจารณาอุปกรณ์ที่เราคุ้นเคยกัน เริ่มจากตัวต้านทานเมื่อให้กระแสที่ไหลผ่าน $i(t) = I_m\cos\omega t$ เนื่องจาก $v = iR$ จะได้ว่า
\[
v(t) = Ri(t) = I_mR\cos\omega t = V_m\cos\omega t
\]
ดังนั้น
\begin{lawbox}{เฟสของ $v$ และ $i$ บนตัวต้านทาน}
    เฟสของ $v$ และ $i$ จะ\emph{ตรงกัน}บนตัวต้านทาน $R$ โดยที่
    \begin{equation}
        V_m = RI_m\label{resac}
    \end{equation}
\end{lawbox}

พิจารณาตัวเก็บประจุเมื่อให้ $v(t) = V_m\cos\omega t$ จะได้ว่า
\[
i(t) = C\odv{v}{t} = CV_m\odv{}{t}\cos\omega t = -\omega CV_m\sin\omega t = I_m\cos\ab(\omega t + \pi/2)
\]
ดังนั้น
\begin{lawbox}{เฟสของ $v$ และ $i$ บนตัวเก็บประจุ}
    เฟสของ $i$ จะนำ $v$ อยู่ $\pi/2$ บนตัวเก็บประจุ $C$ โดยที่
    \begin{equation}
        V_m = \ab(\frac{1}{\omega C})I_m\label{capac}
    \end{equation}
\end{lawbox}

สุดท้าย พิจารณาตัวเหนี่ยวนำเมื่อให้ $i(t) = I_m\cos\omega t$ จะได้ว่า
\[
v(t) = L\odv{i}{t} = LI_m\odv{}{t}\cos\omega t = -\omega LI_m\sin\omega t = V_m\cos\ab(\omega t + \pi/2)
\]
ดังนั้น
\begin{lawbox}{เฟสของ $v$ และ $i$ บนตัวเหนี่ยวนำ}
    เฟสของ $v$ จะนำ $i$ อยู่ $\pi/2$ บนตัวเหนี่ยวนำ $L$ โดยที่
    \begin{equation}
        V_m = \ab(\omega L)I_m\label{indac}
    \end{equation}
\end{lawbox}

สังเกตว่าสมการ (\ref{resac}) ถึง (\ref{indac}) หน้าตาคล้ายกับกฎของ Ohm เราจึงนิยามค่าความต้านทานเสมือนของตัวเก็บประจุและตัวเหนี่ยวนำว่า \emph{รีแอคแตนซ์เชิงประจุ}และ\emph{รีแอคแตนซ์เชิงเหนี่ยวนำ} (\emph{capacitive}/\emph{inductive reactance}: $X_C$ และ $X_L$) ดังนี้:
\begin{defbox}{Reactance}
    \begin{equation}
        X_C \equiv \frac{1}{\omega C}\qq{และ}X_L \equiv \omega L
    \end{equation}
\end{defbox}

\subsection{เฟสเซอร์}

เราสามารถแทนฟังก์ชันไซน์ด้วยส่วนจริงของฟังก์ชัน exponential เชิงซ้อน:
\[
a(t) = A_m\cos\ab(\omega t + \phi) = \Re(A_me^{j\phi}\cex) \equiv \Re(A\cex)
\]
(ในการวิเคราะห์วงจรเราจะใช้ $j$ แทนหน่วยจินตภาพเพราะ $i$ ซ้ำกับกระแส) โดยจำนวนเชิงซ้อน $A$ เรียกว่าเป็นรูปใน\emph{โดเมนเฟสเซอร์/ความถี่} (\emph{phasor/frequency domain}) ของ $a(t)$ (โดยจะเรียก $a(t)$ ว่ารูป\emph{โดเมนเวลา}หรือ \emph{time domain}) โดยเราจะแทน $A = A_me^{j\phi}$ ด้วยสัญลักษณ์
\begin{defbox}{สัญลักษณ์แทนจำนวนเชิงซ้อนในการวิเคราะห์วงจร}
    \begin{equation}
        A_m\phase{\phi} \equiv A_me^{j\phi} = A 
    \end{equation}
\end{defbox}

ต่อมาเรามาดู ``กฎของ Ohm" ของแต่ละอุปกรณ์ใน frequency domain, จาก (\ref{resac}) ถึง (\ref{indac}) ก็จะไดh
\begin{lawbox}{อุปกรณ์ต่าง ๆ ใน Frequency Domain}
    สำหรับตัวต้านทาน $R$:
    \begin{equation}
        V = RI
    \end{equation}
    สำหรับตัวเก็บประจุ $C$:
    \begin{equation}
        V = \ab(\frac{1}{j\omega C})I
    \end{equation}
    สำหรับตัวเหนี่ยวนำ $L$:
    \begin{equation}
        V = \ab(j\omega L)I
    \end{equation}
\end{lawbox}
เราจึงจะนิยามค่า\emph{อิมพีแดนซ์} (\emph{impedance}: $Z$) และ\emph{แอดมิตแทนซ์} (\emph{admittance}: $Y$) ว่า
\begin{defbox}{ Impedance และ Admittance}
    \begin{equation}
        Z\equiv \frac{V}{I}\qq{และ} Y\equiv\frac{1}{Z}=\frac{I}{V}
    \end{equation}
\end{defbox}
ก็จะได้ impedance ของอุปกรณ์ต่าง ๆ:
\begin{eqbox}{Impedance ของอุปกรณ์ต่าง ๆ}
    Z_R = R\quad\quad Z_L = j\omega L \quad\quad Z_C = \frac{1}{j\omega C}
\end{eqbox}
และ admittance ของอุปกรณ์ต่าง ๆ:
\begin{eqbox}{Admittance ของอุปกรณ์ต่าง ๆ}
    Y_R = \frac{1}{R}\quad\quad Y_L = \frac{1}{j\omega L} \quad\quad Y_C = j\omega C
\end{eqbox}

เนื่องจากฟังก์ชัน $\Re$ และ $\Im$ มีสมบัติเชิงเส้น การแทนฟังก์ชันรูปไซน์ใน frequency domain จึงมีประโยชน์ต่อการคำนวณ เพราะกฎที่สำคัญต่าง ๆ ใน time domain ยังคงเป็นจริงใน frequency domain โดยที่สำคัญสุดเลยก็คือกฎของ Kirchhoff:
\begin{ieqbox}{KCL ใน Frequency Domain}
    \sum I_\txt{in} = \sum I_\txt{out}
\end{ieqbox}
และ
\begin{ieqbox}{KVL ใน Frequency Domain}
    \sum_\txt{loop} V = 0
\end{ieqbox}

\subsection{การรวม Impedance}

พิจารณาการต่อวงจรแบบอนุกรม โดยมีอุปกรณ์อยู่สองชิ้นที่มี impedance $Z_1$ และ $Z_2$ จะได้ว่ากระแส $I$ ที่ไหลผ่านอุปกรณ์ทั้งสองต้องเท่ากัน ดังนั้น
\begin{align*}
    v_\txt{total}(t) &= \Re(V_1\cex) + \Re(V_2\cex)\\
    &= \Re(IZ_1\cex) + \Re(IZ_2\cex)\\
    &= \Re\ab(I(Z_1+Z_2)\cex) 
\end{align*}
ดังนั้น $V_\txt{total} = I(Z_1+Z_2)$ แล้วก็จะได้
\[
Z_\txt{total} = Z_1 + Z_2
\]
ขยายเป็น $n$ อุปกรณ์จะได้
\begin{ieqbox}{การรวม Impedance แบบอนุกรม}
    Z_\txt{total} = Z_1 + Z_2 + \dots Z_n
\end{ieqbox}

ต่อมาพิจารณาการต่อวงจรแบบขนาน โดยมีอุปกรณ์สองชิ้นเดิม จะได้ว่าความต่างศักย์ $V$ คร่อมอุปกรณ์ทั้งสองต้องเท่ากัน ดังนั้น
\begin{align*}
    i_\txt{total}(t) &= \Re(I_1\cex) + \Re(I_2\cex)\\
    &= \Re\ab(\frac{V}{Z_1}\cex) + \Re\ab(\frac{V}{Z_2}\cex)\\
    &= \Re\ab(V\ab(\frac{1}{Z_1} + \frac{1}{Z_2})\cex)
\end{align*}
ดังนั้น $I_\txt{total} = V\ab(\frac{1}{Z_1} + \frac{1}{Z_2})$ และจะได้
\[
Y_\txt{total} = \frac{1}{Z_\txt{total}} = \frac{1}{Z_1} + \frac{1}{Z_2}
\]
ขยายเป็น $n$ อุปกรณ์จะได้
\begin{ieqbox}{การรวม Impedance แบบขนาน}
    \frac{1}{Z_\txt{total}} = \frac{1}{Z_1} + \frac{1}{Z_2} + \dots + \frac{1}{Z_n}
\end{ieqbox}
หรือ
\begin{eqbox}{การรวม Admittance แบบขนาน}
    Y_\txt{total} = Y_1 + Y_2 + \dots Y_n
\end{eqbox}

\subsection{Resonance ในวงจรไฟฟ้ากระแสสลับ}







