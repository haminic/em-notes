\chapter{พลศาสตร์ไฟฟ้า}
\section{แรงเคลื่อนไฟฟ้า}
\subsection{กฎของ Ohm}

ในการเคลื่อนย้ายประจุให้เกิดกระแสก็จะต้องออกแรง เราจึงมาหาความสัมพันธ์ระหว่างแรงกับกระแสกันก่อน

พิจารณาสายไฟที่มีอิเล็กตรอนอิสระอยู่ $n$ อนุภาคต่อหน่วยปริมาตรและแต่ละอิเล็กตรอนมีมวล $m$ ประจุ $q$ และสมมติมีสนามแรง $\vb{f}$ (ต่อหน่วยประจุ) กระทำอยู่กับทั้งสาย แรง $\vb{f}$ จะทำให้อิเล็กตรอนเคลื่อนที่ด้วยอัตราเร่ง $a$ ก่อนที่จะชนกับอิเล็กตรอนอีกอนุภาคจนทำให้อัตราเร็ว (โดยเฉลี่ยทั้งหมดแล้ว) กลับมาเป็น $0$ อีกครั้ง โดยถ้าสมมติว่าอัตราเร็วของอิเล็กตรอนเนื่องจากความร้อนเท่ากับ $v_\txt{thermal}$ และมีระยะทางเฉลี่ย $\lambda$ ระหว่างการชน เนื่องจาก $v_\txt{thermal}$ มีค่าสูงมาก จึงประมาณได้ว่าความเร่งที่เกิดขึ้นนั้นมีผลน้อยมาก จึงได้เวลาโดยเฉลี่ยก่อนที่จะชนกับอิเล็กตรอนอีกอนุภาคคือ
\[
t = \frac{\lambda}{v_\txt{thermal}}
\]
ก็จะได้ความเร็วเฉลี่ยหรือ\emph{อัตราเร็วลอยเลื่อน} (\emph{drift velocity}) เท่ากับ
\[
v_d = \frac{1}{2}at = \frac{a\lambda}{2v_\txt{thermal}}
\]
ดังนั้นกระแสจึงเท่ากับ
\begin{equation}
    \vb{J} = nq\vb{v}_d = nq\frac{\lambda\vb{a}}{2v_\txt{thermal}} = \ab(\frac{nq\lambda}{2v_\txt{thermal}m})\vb{F} = \ab(\frac{n^2q^2\lambda}{2v_\txt{thermal}m})\vb{f}\label{ohmapprox}
\end{equation}
จะเห็นว่าโดยปกติแล้วสำหรับวัสดุทั่วไป $\vb{J}$ จึงแปรผันตรงกับ $\vb{f}$:
\begin{eqbox}{สมการการแปรผันตรงของกระแสกับแรง}
    \vb{J} = \sigma\vb{f}
\end{eqbox}
โดยที่ $\sigma$ เป็นค่าคงที่ที่เรียกว่า\emph{สภาพนำไฟฟ้า} (\emph{conductivity}) ของสสารนั้น (ถ้าสสารเป็นตัวนำในอุดมคติก็จะมี $\sigma = \infty$) และ $\rho\equiv 1/\sigma$ เรียกว่า\emph{สภาพต้านทาน} (\emph{resistivity}) โดยถ้าแรงที่ใช้เป็นแรงทางไฟฟ้า\underline{เท่านั้น}โดยมีส่วนของแรงแม่เหล็กน้อยมาก ๆ ก็จะได้
\begin{ieqbox}{กฎของ Ohm}
    \vb{J} = \sigma\vb{E}\label{ohme}
\end{ieqbox}
หมายเหตุ: \emph{สมการ (\ref{ohmapprox}) เป็นเพียงการประมาณหยาบ ๆ เท่านั้น จึงไม่สามารถนำมาใช้หา $\sigma$ ได้จริง ๆ และยิ่งไปกว่านั้น ในความเป็นจริงแล้วยังมีวัสดุบางชนิดที่ไม่เป็นไปตามกฎการแปรผันตรงนี้อีกด้วย เราจะเรียกวัสดุที่เป็นไปตามกฎของ Ohm ว่าเป็นวัสดุ Ohmic}

สังเกตว่าในการทำให้ความต่างศักย์มากขึ้น $k$ เท่าระหว่างขั้ว\emph{อิเล็กโทรด} เราจะต้องเพิ่ม $Q$ ไป $k$ เท่า ทำให้ $\vb{E}$ เพิ่ม $k$ เท่าและจาก (\ref{ohme}) จะได้ว่า $\vb{J}$ และ $I$ ก็เพิ่ม $k$ เท่าเช่นกัน ก็จะได้กฎของ Ohm ในอีกรูปแบบ:
\begin{ieqbox}{กฎของ Ohm ในรูปกระแสและความต่างศักย์}
    V = IR
\end{ieqbox}
เมื่อ $R$ เป็นค่าคงที่\emph{ความต้านทาน}ระหว่างสองจุดนั้น (ในการคำนวณหาความต้านทานใช้ (\ref{ohme}) ตามในแต่ละระบบได้เลย)

ในกรณีที่กระแสไหลแบบคงที่ในสสารเนื้อเดียวกันที่เป็นไปตามกฎของ Ohm จาก (\ref{gdjzero}) จะได้ว่า
\begin{equation}
    \gd\cdot\vb{E} = \frac{1}{\sigma}\gd\cdot\vb{J} = 0
\end{equation}
ดังนั้นในบริเวณที่สสารเป็นไปตามกฎของ Ohm ก็จะไม่มีประจุตกค้างอยู่ภายในเลย จึงทำให้สามารถใช้ทริคในการแก้ศักย์และสนามจากสมการ Laplace ได้ตามปกติ

สุดท้าย จาก (\ref{ohmapprox}) เนื่องจากแรงที่ออกนั้นไม่ส่งผลในอัตราเร็วลอยเลื่อนเพิ่มขึ้นเลย ดังนั้นพลังงานส่วนมากจากการชนจะถูกเปลี่ยนเป็นความร้อน โดยถ้ามีประจุไหลต่อเวลาเท่ากับ $I$ โดยศักย์ของประจุลดลง $V$ ก็จะได้
\begin{ieqbox}{กฎการให้ความร้อนของ Joule}
    P = IV = I^2R = \frac{V^2}{R}
\end{ieqbox}

\subsection{แรงเคลื่อนไฟฟ้า}

โดยปกติแล้วในวงจรไฟฟ้าจะมีแรงสองแรงในการทำให้ประจุเคลื่อนที่คือแรงจากแหล่งกำเนิด ($\vb{f}_s$) ซึ่งโดยปกติแล้วแรงนี้จะอยู่แค่ในบริเวณแหล่งกำเนิดเท่านั้น และอีกแรงคือแรงจากสนามไฟฟ้าที่จะเป็นตัวที่ช่วยทำให้กระแสไหลด้วย $I$ คงที่ตลอดทั้งสาย (เหตุผลที่ $I$ ต้องคงที่เพราะถ้าไม่คงที่แล้วจะมีส่วนของสายไฟที่เกิดการรวมตัวกันของประจุ ทำให้เกิด $\vb{E}$ ที่ต้านกระแสในส่วนที่เร็วเกิน และส่งเสริมส่วนที่ไหลช้าให้ไหลเร็วขึ้น) ดังนั้นแรงต่อประจุโดยรวมจะเท่ากับ
\[
\vb{f} = \vb{f}_s + \vb{E}
\]
เราจึงนิยามผลของแรงทั้งหมดภายในวงจรว่า\emph{แรงเคลื่อนไฟฟ้า}หรือ \emph{emf} (\emph{electromotive force}: $\emf$):
\begin{defbox}{แรงเคลื่อนไฟฟ้า}
    \begin{equation}
        \emf \equiv \oint \vb{f}\cdot\odif{\vbs{\ell}} = \oint \vb{f}_s\cdot\odif{\vbs{\ell}}
    \end{equation}
\end{defbox}
เนื่องจาก $\oint \vb{E}\cdot\odif{\vbs{\ell}} = 0$

สมมติแหล่งกำเนิดเป็นแบตเตอรี่ไร้ความด้านทาน ($\sigma = \infty$) ก็จะได้ว่าแรงที่ออกในการเคลื่อนประจุเป็น $0$ ดังนั้น $0 = \vb{f} = \vb{f}_s + \vb{E}$ ก็จะได้
\begin{equation}
    V = -\int_{\vb{a}}^{\vb{b}} \vb{E}\cdot\odif{\vbs{\ell}} = \int_{\vb{a}}^{\vb{b}} \vb{f}_s\cdot\odif{\vbs{\ell}} = \oint \vb{f}_s\cdot\odif{\vbs{\ell}} = \emf
\end{equation}
แต่ถ้าแบตเตอรี่นี้มีความต้านทาน $r$ (หมายความว่าถ้าตัดแรง $\vb{f}_s$ ออกแล้วความต่างศักย์ $V_\txt{off} = \int\vb{E}_\txt{off}\cdot\odif{\vbs{\ell}} = Ir$) สมการด้านบนจะไม่เป็นจริง โดยจะได้
\begin{equation}
    V = -\int_{\vb{a}}^{\vb{b}} \vb{E}\cdot\odif{\vbs{\ell}} = \int_{\vb{a}}^{\vb{b}} \ab(\vb{f}_s - \frac{\vb{J}}{\sigma})\cdot\odif{\vbs{\ell}} = \emf + \int_{\vb{a}}^{\vb{b}} \vb{E}_\txt{off}\cdot\odif{\vbs{\ell}} = \emf - V_\txt{off} = \emf - Ir
\end{equation}