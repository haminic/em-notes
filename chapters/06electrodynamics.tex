\chapter{พลศาสตร์ไฟฟ้า}
\section{แรงเคลื่อนไฟฟ้า}
\subsection{กฎของ Ohm}

ในการเคลื่อนย้ายประจุให้เกิดกระแสก็จะต้องออกแรง โดยปกติแล้วสำหรับสสารทั่วไป แรงที่ออกจะแปรฝันตรงกับกระแส:
\begin{eqbox}{สมการการแปรผันตรงของกระแสกับแรง}
    \vb{f} = \sigma\vb{E}
\end{eqbox}
โดยที่ $\sigma$ เป็นค่าคงที่ที่เรียกว่า\emph{สภาพนำไฟฟ้า} (\emph{conductivity}) ของสสารนั้น และ $\rho\equiv 1/\sigma$ เรียกว่า\emph{สภาพต้านทาน} (\emph{resistivity}) โดยถ้าแรงที่ใช้เป็นแรงทางไฟฟ้าที่มีส่วนแรงแม่เหล็กน้อยมาก ๆ ก็จะได้
\begin{ieqbox}{กฎของ Ohm}
    \vb{J} = \sigma\vb{E}
\end{ieqbox}