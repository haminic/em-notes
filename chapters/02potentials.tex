\chapter{ศักย์ไฟฟ้า}
\section{สมการ Laplace}
\subsection{สมการ Laplace ในสามมิติ}
ในการแก้หาสนามไฟฟ้า ถ้าไม่มีความสมมาตรพอที่จะใช้กฎของ Gauss (\ref{gauss}) อาจจะง่ายกว่าที่จะหาศักย์ไฟฟ้าก่อน โดยเรามักสนใจศักย์ไฟฟ้าในบริเวณที่ไม่ได้อยู่ในเนื้อประจุ ดังนั้นสมการ Laplace (\ref{laplaceeq}) จึงเป็นสมการที่สำคัญ โดยมีสมบัติของผลเฉลยของมัน (ซึ่งเรียกว่า\emph{ฟังก์ชันฮาร์มอนิก}) ที่ควรรู้คือ
\begin{lawbox}{สมบัติของผลเฉลยของสมการ Laplace ในสามมิติ}
    ถ้า $V$ เป็นผลเฉลยของสมการ Laplace แล้ว
    \begin{compactenum}
        \item $V$ มีค่าเท่ากับค่าเฉลี่ยของ $V$ รอบ ๆ หรือก็คือ สำหรับทุก $\vb{r}$ และพื้นผิวทรงกลม $\sur$ รัศมี $R$ ที่มีจุดศูนย์กลางที่ $\vb{r}$ จะได้ว่า
        \begin{equation}
            V(\vb{r})=\frac{1}{4\pi R^2}\oint_\sur V\odif{a}
        \end{equation}
        \item $V$ ไม่มีค่าสุดขีดสัมพัทธ์ นั่นคือค่าสุดขีดทั้งหมดของ $V$ ในปริมาตร $\vol$ จะอยู่บน $\del\vol$ เท่านั้น
    \end{compactenum}
\end{lawbox}
หมายเหตุ: \emph{ทฤษฎีบทต่าง ๆ เกี่ยวกับสมการ Laplace มักจะใช้ได้เมื่อปริมาตร $\vol$ ที่สนใจนั้นมี $\rho=0$ เท่านั้น ดังนั้นต้องเลือกปริมาตรดี ๆ}
\begin{proof}
    ให้จุดประจุ $q$ อยู่ที่ $(0,0,z)$ พิจารณาค่าเฉลี่ยของ $V$ บนทรงกลมที่อยู่ที่จุดกำเนิดที่มีรัศมี $R$ (ให้ $\theta$ เป็นมุมที่ $\vb{r}$ ทำกับแกน $+z$)
    {\allowdisplaybreaks
    \begin{align*}
        \frac{1}{4\pi R^2}\oint_\sur V\odif{a}&=\frac{1}{4\pi R^2}\oint_S\kem\frac{q}{\rad}\odif{a}\\
        &=\frac{1}{4\pi R^2}\frac{q}{4\pi\eps_0}\int_0^\pi\int_0^{2\pi}\frac{1}{\sqrt{z^2+R^2-2zR\cos\theta}}R^2\sin\theta\odif{\phi}\odif{\theta}\\
        &=\frac{1}{2}\frac{q}{4\pi\eps_0}\int_0^\pi\frac{1}{\sqrt{z^2+R^2-2zR\cos\theta}}R^2\sin\theta\odif{\theta}\\
        &=\frac{1}{2}\frac{q}{4\pi\eps_0}\frac{1}{zR}\eval{\sqrt{z^2+R^2-2zR\cos\theta}}_0^\pi\\
        &=\frac{1}{2}\frac{q}{4\pi\eps_0}\frac{1}{zR}\ab\big(\ab(z+R)-\ab(z-R))\\
        &=\kem\frac{q}{z}\\
        &=V(\vb{0})
    \end{align*}
    }
    ซึ่งเป็นไปตามต้องการสำหรับจุดประจุ ดังนั้นจึงเป็นจริงสำหรับสนามใด ๆ ก็ตาม
    
    ส่วนข้อ 2. ได้มาจากข้อ 1. โดยตรง เพราะถ้าค่าใด ๆ ของ $V$ เกิดจากค่าเฉลี่ยของจุดรอบ ๆ ค่า $V$ ค่านั้นไม่มีทางเป็นค่าสุดขีดสัมพัทธ์
\end{proof}
\subsection{Uniqueness ของผลเฉลยของสมการ Laplace}
\begin{lawbox}{ทฤษฎีบท Uniqueness ที่หนึ่ง}
    สมการ Laplace จะมีผลเฉลยเดียวบนปริมาตร $\vol$ ถ้ารู้ค่า $V$ ทั้งหมดบน $\del\vol$
\end{lawbox}
\begin{proof}
    ให้ $V_1$ และ $V_2$ เป็นผลเฉลยของสมการ Laplace บนปริมาตร $\vol$ ที่มีค่าตรงกันบน $\del\vol$ ดังนั้น
    \[
    V_3\equiv V_1-V_2
    \]
    เป็นผลเฉลยของสมการ Laplace ที่มีค่าที่ $\del\vol$ เท่ากับ $0$\\
    แต่เนื่องจากค่าสุดขีดของสมการ Laplace จะต้องอยู่บน $\del\vol$ ดังนั้น $V_3=0$ ทุกที่ในปริมาตร หรือก็คือ
    \[
    V_1=V_2
    \]
    ตามต้องการ
\end{proof}
และไม่ยากที่จะขยายทฤษฎีบทนี้กับสมการ Poisson โดยใช้วิธีพิสูจน์คล้าย ๆ กับด้านบนจะได้ว่า:
\begin{corbox}{บทตั้ง}
    บนปริมาตร $\vol$ ถ้ารู้ $\rho$ ภายในปริมาตรและรู้ค่า $V$ ทั้งหมดบน $\del\vol$ แล้วจะได้ว่ามีสนาม $V$ ในปริมาตรนั้นที่สอดคล้องกับเงื่อนไขเพียงสนามเดียว
\end{corbox}

\begin{lawbox}{ทฤษฎีบท Uniqueness ที่สอง}
    บนปริมาตร $\vol$ ที่มีขอบเขตอยู่บนผิวของตัวนำ (อาจมีขอบเขตหนึ่งเป็นตัวนำที่ $\infty$ ได้) ถ้ารู้ค่า $\rho$ ภายในปริมาตรและรู้ค่า $Q$ ของตัวนำทั้งหมดแล้วจะได้ว่ามีสนาม $\vb{E}$ ในปริมาตรนั้นที่สอดคล้องกับเงื่อนไขทั้งหมดเพียงสนามเดียว
\end{lawbox}
\begin{proof}
    ให้ $\vb{E}_1$ และ $\vb{E}_2$ เป็นสนามใน $\vol$ ที่สอดคล้องกับเงื่อนไข และให้ $\vb{E}_3=\vb{E}_1-\vb{E}_2$ จาก (\ref{gaussde}) จะได้ว่า
    \begin{equation}
        \gd\cdot\vb{E}_3=0\tag{$*$1}
    \end{equation}
    และจาก (\ref{gauss}) จะได้ว่า
    \begin{equation}
        \oint\vb{E}_3\cdot\odif{\vb{a}}=0\tag{$*$2}\label{u2pf2}
    \end{equation}
    สำหรับทุก ``\emph{ผิวย่อย}" ของ $\del\vol$ ต่อมาพิจารณา
    \[
    \gd\cdot(V_3\vb{E}_3)=\gd V_3\cdot\vb{E}_3+V_3(\cancel{\gd\cdot\vb{E}_3})=-E_3^2
    \]
    และโดย divergence theorem จะได้ว่า
    \begin{equation}
    \oint_{\del\vol}V_3\vb{E}_3\cdot\odif{\vb{a}}=\int_\vol\gd\cdot(V_3\vb{E}_3)\odif{\tau}=-\int_\vol E_3^2\odif{\tau}\tag{$*$3}\label{u2pf3}
    \end{equation}
    แต่เนื่องจากทุกผิวย่อยของ $\del\vol$ บนแต่ละตัวนำมี $V_3$ คงที่จะได้ว่า
    \[
    \oint_{\del\vol}V_3\vb{E}_3\cdot\odif{\vb{a}}=\sum_{\txt{\sur}}V_\sur\oint_\sur\vb{E}_3\cdot\odif{\vb{a}}\overset{\text{(\ref{u2pf2})}}{=}0
    \]
    นำไปใส่กลับใน (\ref{u2pf3}) จะได้ว่า $\int_\vol E_3^2\odif{\tau}=0$ ดังนั้น $\vb{E}_3=\vb{0}$ หรือก็คือ
    \[
    \vb{E}_1=\vb{E}_2
    \]
    ตามต้องการ
\end{proof}

\section{การจำลองภาพ}
\subsection{การสร้างระบบใหม่เพื่อแก้หาสนาม}
ในบางครั้งการหาศักย์ไฟฟ้าตรง ๆ อาจจะยาก แต่ถ้าหาระบบใหม่ที่มีค่า $V$ ที่บริเวณขอบเขตและ $\rho$ ตรงกับค่าบนระบบที่เราสนใจ จากทฤษฎีบท uniqueness ที่หนึ่ง จะได้ว่าศักย์ไฟฟ้าในบริเวณที่สนใจของทั้งสองระบบจะเท่ากันพอดี ยกตัวอย่างเช่น

\begin{corbox}{ตัวอย่าง}
    ในระบบพิกัดฉากสามมิติ มีแผ่นตัวนำที่ต่อสายดินวางอยู่ทั่วทั้งระนาบ $xy$ และมีจุดประจุ $q$ วางอยู่ ณ จุด $(0,0,d)$ จงหาศักย์ไฟฟ้าในบริเวณด้านบนแผ่นตัวนำ
\end{corbox}
\begin{soln}
    พิจารณาอีกระบบที่มีจุดประจุ $q$ ที่ $(0,0,d)$ และ $-q$ ที่ $(0,0,-d)$ สังเกตว่าระบบนี้มีสภาวะขอบเขตของศักย์ไฟฟ้าในปริมาตรเหนือระนาบ $xy$ ตรงกันกับระบบในโจทย์เลย ($V=0$ บนระนาบ $xy$, $V=0$ ที่บริเวณไกลมาก ๆ) ดังนั้นโดยทฤษฎีบท uniqueness ที่หนึ่ง ทั้งสองระบบนี้จะต้องมีสนามศักย์ไฟฟ้าตรงกันบนปริมาตรเหนือระนาบ $xy$ ดังนั้นจึงได้ว่า
    \[
    V(x,y,z)=\begin{cases}
        \dfrac{q}{4\pi\eps_0}\ab(\ab(x^2+y^2+(z-d)^2)^{-1/2}-\ab(x^2+y^2+(z+d)^2)^{-1/2})&\qq{เมื่อ $z\geq0$}\\
        0&\qq{เมื่อ $z< 0$}
    \end{cases}
    \]
    ($V(x,y,z)=0$ เมื่อ $z<0$ เพราะด้านล่างเหมือนกับระบบที่ไม่มีประจุที่ใดเลย)
\end{soln}
หมายเหตุ: \emph{ควรระวังว่าระบบที่สร้างขึ้นมาเปรียบเทียบนี้จะต้องมีการกระจายตัวของประจุในบริเวณที่สนใจเหมือนกับระบบตั้งต้นเท่านั้นจึงจะใช้ได้ และไม่ได้แปลว่าทุกอย่างของทั้งสองระบบจะเหมือนกัน เช่น ถ้าลองคำนวณดูแล้วพลังงานของระบบโจทย์จะเป็นครึ่งหนึ่งของระบบที่สร้างขึ้นมาใหม่ (มาจากสนามอีกครึ่งที่หายไป)}
\section{การแยกตัวแปร}
\subsection{การแยกตัวแปรบนพิกัดคาร์ทีเซียน}
เริ่มจากการ ``เดา" ว่า 
\[
V(x,y,z)=X(x)Y(y)Z(z)
\]
ดังนั้นจากสมการ Laplace จะได้ว่า
\begin{align*}
    YZ\odv[2]{X}{x}+XZ\odv[2]{Y}{y}+XY\odv[2]{Z}{z}&=0\\
    \frac{1}{X}\odv[2]{X}{x}+\frac{1}{Y}\odv[2]{Y}{y}+\frac{1}{Z}\odv[2]{Z}{z}&=0
\end{align*}
เนื่องจากแต่ละพจน์เป็นฟังก์ชันตัวแปรเดียวโดยต้องรวมกันเท่ากับ $0$ ทุก $(x,y,z)$ ในปริมาตรที่สนใจ ดังนั้น
\[
\frac{1}{X}\odv[2]{X}{x}=C_x\qq*{,}\frac{1}{Y}\odv[2]{Y}{y}=C_y\qq*{,}\frac{1}{Z}\odv[2]{Z}{z}=C_z
\]
จากนั้นใช้เงื่อนไขขอบเขตในโจทย์เพื่อดูว่า $C$ ในแต่ละสมการควรเป็นค่าบวกหรือลบ และแก้สมการเชิงอนุพันธ์ออกมาโดยจะมีคำตอบดังนี้:
\begin{lawbox}{สมการเชิงอนุพันธ์ของสมการ Laplace ในพิกัดคาร์ทีเซียน}
    สมการเชิงอนุพันธ์
    \begin{equation}
        \odv[2]{T}{t}=CT
    \end{equation}
    มีคำตอบคือ
    \begin{equation}
        \begin{cases}
            Ae^{kt}+Be^{-kt}&\qq{ถ้า $C=k^2>0$}\\
            At+B&\qq{ถ้า $C=0$}\\
            A\sin kt+B\cos kt&\qq{ถ้า $C=-k^2<0$}
        \end{cases}
    \end{equation}
    เมื่อ $A$ และ $B$ เป็นค่าคงที่
\end{lawbox}
จากนั้นแก้หาค่าคงที่ให้ได้มากที่สุดเท่าที่เป็นไปได้จากเงื่อนไขโจทย์ จะได้เซตของผลเฉลยมาเซตหนึ่งที่อาจไม่มีผลเฉลยใดเลยสอดคล้องกับเงื่อนไขขอบเขตของโจทย์ เนื่องจากสมการ Laplace เป็นสมการเชิงเส้น ดังนั้นเราอาจจะหาวิธีการนำผลเฉลยที่ได้จากการแยกตัวแปรนี้มาบวกกันให้ได้คำตอบที่ตรงกับค่าขอบเขตได้ ซึ่งผลเฉลยเหล่านี้ในกรณีนี้จะอยู่ในรูป $\sin$ จึงสามารถใช้การวิเคราะห์ Fourier เพื่อนำผลเฉลยมาบวกกันให้ได้ค่าที่ตรงกับค่าขอบเขต โดยเราจะหาสัมประสิทธิ์ของแต่ละพจน์ในอนุกรม Fourier ได้โดยใช้ทริคดังต่อไปนี้
\begin{eqbox}{อินทิกรัลสำคัญในการวิเคราะห์ Fourier}
    \int_0^\pi\sin\ab(nt)\sin\ab(n't)\odif{t}=\begin{cases}
        0&\qq{ถ้า $n'\neq n$}\\
        \dfrac{\pi}{2}&\qq{ถ้า $n'=n$}
    \end{cases}
\end{eqbox}
หรือแทนตัวแปร $t\mapsto(\pi/a)t$ ได้เป็น
\begin{equation}
    \int_0^a\sin\ab(\frac{n\pi t}{a})\sin\ab(\frac{n'\pi t}{a})\odif{t}=\begin{cases}
        0&\qq{ถ้า $n'\neq n$}\\
        \dfrac{a}{2}&\qq{ถ้า $n'=n$}
    \end{cases}\label{fourierint}
\end{equation}
ดังนั้นถ้าต้องการหาสัมประสิทธิ์ของพจน์ที่ $n$ ที่ทำให้อนุกรม Fourier เท่ากับฟังก์ชัน $V(x)$ ฝั่งซ้าย:
\[
V(x)=\sum_{n=0}^\infty c_n\sin\ab(\frac{n\pi x}{a})
\]
สามารถคูณ $\sin\ab(n\pi x/a)$ เข้าไปทั้งสองฝั่งแล้วอินทิเกรตโดยใช้ (\ref{fourierint}) จะได้
\begin{equation}
    C_n=\int_0^a V(x)\sin\ab(\frac{n\pi x}{a})\odif{x}
\end{equation}
เหตุผลที่เราสามารถทำแบบนี้กับเซตของฟังก์ชัน $\sin$ เหล่านั้นได้เป็นเพราะ
\begin{enumerate}
    \item เซตของฟังก์ชันนี้เป็นเซตที่\emph{สมบูรณ์} (\emph{complete}) หมายความว่า ฟังก์ชันใด ๆ สามารถถูกเขียนได้ในรูปผลบวกเชิงเส้นของฟังก์ชันในเซต
    \item เซตของฟังก์ชันนี้ (ให้เป็น $\{f_1,f_2,f_3,\dots\}$) เป็นเซตที่\emph{ตั้งฉากกัน} (\emph{orthogonal}) หมายความว่า
    \[
    \int f_n(t)\,f_{n'}(t)\odif{t}=0
    \]
    สำหรับทุก $n'\neq n$
\end{enumerate}
\subsection{การแยกตัวแปรบนพิกัดทรงกลม}
ในส่วนนี้จะพิจารณาแค่ระบบที่มีความสมมาตรแบบ azimuth (สมมาตรรอบแกน $z$) ดังนั้นให้
\[
V(r,\theta,\phi)=R(r)\,\Theta(\theta)
\]
จากสมการ Laplace (ในระบบพิกัดทรงกลม) จะได้ว่า
\begin{align*}
    \Theta\odv{}{r}\ab(r^2\odv{R}{r})+\frac{R}{\sin\theta}\odv{}{\theta}\ab(\sin\theta\odv{\Theta}{\theta})&=0\\
    \frac{1}{R}\odv{}{r}\ab(r^2\odv{R}{r})+\frac{1}{\Theta\sin\theta}\odv{}{\theta}\ab(\sin\theta\odv{\Theta}{\theta})&=0
\end{align*}
เช่นเดียวกับในพิกัดคาร์ทีเซียน แต่ละพจน์จะต้องเป็นค่าคงที่ ดังนั้น
\[
\frac{1}{R}\odv{}{r}\ab(r^2\odv{R}{r})=C_r\qq*{,}\frac{1}{\Theta\sin\theta}\odv{}{\theta}\ab(\sin\theta\odv{\Theta}{\theta})=C_\theta
\]
เมื่อให้ $C_r=l(l+1)$ และ $C_\theta=-l(l+1)$ จะแก้สมการได้คำตอบดังนี้:
\begin{lawbox}{สมการเชิงอนุพันธ์ของสมการ Laplace ในพิกัดทรงกลม 1}
    สมการเชิงอนุพันธ์
    \begin{equation}
        \odv{}{r}\ab(r^2\odv{R}{r})=l(l+1)R
    \end{equation}
    มีคำตอบคือ
    \begin{equation}
        R(r)=Ar^l+\frac{B}{r^{l+1}}\label{sphere1}
    \end{equation}
    เมื่อ $A$ และ $B$ คือค่าคงที่
\end{lawbox}
แต่อีกสมการหนึ่งจะยากหน่อย:
\begin{lawbox}{สมการเชิงอนุพันธ์ของสมการ Laplace ในพิกัดทรงกลม 2}
    สมการเชิงอนุพันธ์
    \begin{equation}
        \odv{}{\theta}\ab(\sin\theta\odv{\Theta}{\theta})=-l(l+1)\sin\theta\,\Theta
    \end{equation}
    มีคำตอบคือ
    \begin{equation}
        \Theta(\theta)=A\cdot P_l(\cos\theta)\label{sphere2}
    \end{equation}
    เมื่อ $P_l$ คือพหุนาม Legendre ดีกรี $l$ และ $A$ คือค่าคงที่
\end{lawbox}
หมายเหตุ: \emph{คำตอบในด้านบนเป็นเพียงส่วนเดียวจากคำตอบทั้งหมดเท่านั้น แต่ที่ไม่พิจารณาส่วนของค่าคงที่อีกตัวเพราะส่วนนั้นจะลู่ออกเสมอที่ค่า $\theta$ เท่ากับ $0$ และ $\pi$ (ในกรณีที่บนแกน $z$ ไม่นำมาคิดอาจต้องพิจารณาคำตอบอื่นนี้)}

โดยพหุนาม Legendre หาได้ดังสูตรต่อไปนี้
\begin{eqbox}{สูตรของ Rodrigues}
    P_l(x)=\frac{1}{2^ll!}\odv[l]{}{x}(x^2-1)^l\label{rodrigues}
\end{eqbox}
ดังนั้นในการใช้สูตรนี้จึงจะสมมติว่า $l$ เป็นจำนวนเต็มไม่ลบและแต่ละพหุนามจะมีแค่พจน์กำลังคู่หรือคี่เท่านั้น โดยเมื่อแทนสูตร Rodrigues เข้าไปจะได้พหุนาม Legendre ที่มีดีกรีตั้งแต่ $0$ ถึง $5$ คือ:
{\allowdisplaybreaks
\begin{align*}
    P_0(x)&=1\\
    P_1(x)&=x\\
    P_2(x)&=(3x^2-1)/2\\
    P_3(x)&=(5x^3-3x)/2\\
    P_4(x)&=(35x^4-30x^2+3)/8\\
    P_5(x)&=(63x^5-70x^3+15x)/8
\end{align*}
}

จากนั้นเมื่อแก้ค่าคงที่ออกมามักจะเหลือเซตของผลเฉลยที่เป็นพหุนาม Legendre โดยเซตของพหุนาม Legendre นี้ เช่นเดียวกับ $\sin$ เป็นเซตของฟังก์ชันที่สมบูรณ์และตั้งฉากกันบน $(-1,1)$ โดย
\begin{eqbox}{สมบัติการตั้งฉากกันของพหุนาม Legendre}
    \int_{-1}^1P_l(x)\,P_{l'}(x)\odif{x}=\begin{cases}
        0&\qq{ถ้า $l'\neq l$}\\
        \dfrac{2}{2l+1}&\qq{ถ้า $l'=l$}
    \end{cases}
\end{eqbox}
หรือเมื่อแทนค่า $x=\cos\theta$ จะได้
\begin{equation}
    \int_{0}^\pi P_l(\cos\theta)\,P_{l'}(\cos\theta)\sin\theta\odif{\theta}=\begin{cases}
        0&\qq{ถ้า $l'\neq l$}\\
        \dfrac{2}{2l+1}&\qq{ถ้า $l'=l$}
    \end{cases}
\end{equation}
ซึ่งสามารถใช้ในการแก้หาสัมประสิทธิ์ของคำตอบสุดท้ายที่เป็นการนำคำตอบแบบแยกตัวแปรมาบวกกันได้
\subsection{การแยกตัวแปรบนพิกัดทรงกระบอก}
จะพิจารณาระบบที่สมมาตรแบบทรงกระบอก (สมมาตรในแนวแกน $z$) ดังนั้นให้
\[
V(s,\phi,z)=S(s)\,\Phi(\phi)
\]
จากสมการ Laplace (ในระบบพิกัดทรงกระบอก) จะได้ว่า
\begin{align*}
    \frac{\Phi}{s}\odv{}{s}\ab(s\odv{S}{s})+\frac{S}{s^2}\odv[2]{\Phi}{\phi}&=0\\
    \frac{s}{S}\odv{}{s}\ab(s\odv{S}{s})+\frac{1}{\Phi}\odv[2]{\Phi}{\phi}&=0
\end{align*}
จะได้ว่า
\[
\frac{s}{S}\odv{}{s}\ab(s\odv{S}{s})=C_s\qq*{,}\frac{1}{\Phi}\odv[2]{\Phi}{\phi}=C_\phi
\]
โดยถ้าให้ $C_s=k^2=-C_\phi$ (เพราะถ้า $C_\phi$ ไม่เป็นลบจะได้คำตอบในรูป exponential ทำให้ไม่เป็นฟังก์ชันคาบตามที่ต้องการ) จะได้คำตอบของ $\Phi$ เป็น $\Phi(\phi)=A\sin k\phi+B\cos k\phi$ เช่นเดียวกับในพิกัดคาร์ทีเซียน และ
\begin{lawbox}{สมการเชิงอนุพันธ์ในพิกัดทรงกระบอก}
    สมการเชิงอนุพันธ์
    \begin{equation}
    \odv{}{s}\ab(s\odv{S}{s})=\frac{k^2}{s}S\label{cylde}
    \end{equation}
    มีคำตอบคือ
    \begin{equation}
        S(s)=As^k+Bs^{-k}
    \end{equation}
    เมื่อ $A$ และ $B$ คือค่าคงที่
\end{lawbox}
แต่เมื่อ $k=0$ จะได้คำตอบเดียวคือค่าคงที่ ซึ่งไม่ครบกับอันดับของสมการ (เมื่อนำมารวมกันตอนสุดท้ายอาจทำให้ได้คำตอบไม่ครบได้ แต่กรณีของ $\Phi$ เหตุผลที่ไม่นำ $A\phi+B$ ที่เป็นผลเฉลยในกรณี $k=0$ มาใช้เพราะว่าเห็นชัดว่า $A$ ต้องเป็น $0$ ซึ่งรวมอยู่ในกรณี $k=0$ ของ $A\sin k\phi + B\cos k\phi$ อยู่แล้ว) จึงต้องคิดแยกกรณี:
\begin{lawbox}{กรณี $k=0$}
    สมการ (\ref{cylde}) ถ้า $k=0$ จะได้คำตอบคือ
    \begin{equation}
        S(s)=A\log s+B
    \end{equation}
    เมื่อ $A$ และ $B$ คือค่าคงที่
\end{lawbox}
โดยในการหาสัมประสิทธิ์ของคำตอบต่อไปให้ใช้การวิเคราะห์ Fourier แบบเดียวกับพิกัดคาร์ทีเซียน

\section{การกระจาย Multipole}
\subsection{การประมาณศักย์ไฟฟ้าระยะไกล}
พิจารณา \emph{electric dipole} ที่ประกอบด้วยจุดประจุ $+q$ และ $-q$ ที่ห่างกัน $d$ โดยสมมติให้ dipole นี้ตั้งในแกน $z$ โดยมีประจุบวกอยู่ในทิศ $+z$ และจุดศูนย์กลางของ dipole อยู่ที่จุดกำเนิด และให้ $\vbs{\rad}_+$, $\vbs{\rad}_-$ เป็นเวกเตอร์จากขั้วบวกและลบมายัง $\vb{r}$ ตามลำดับ จะได้ว่า
\[
V(\vb{r})=\kem\ab(\frac{q}{\rad_+}-\frac{q}{\rad_-})
\]
และจากกฎของ $\cos$ จะได้
\[
\rad_\pm^2=r^2+(d/2)^2\mp rd\cos\theta=r^2\ab(1\mp\frac{d}{r}\cos\theta+\frac{d^2}{4r^2})
\]
ดังนั้นเมื่อ $r\gg d$ จะได้ว่า
\[
\frac{1}{\rad_\pm}\approx \frac{1}{r}\ab(1\mp\frac{d}{r}\cos\theta)^{-1/2}\approx\frac{1}{r}\ab(1\pm \frac{d}{2r}\cos\theta)
\]
ก็จะได้ว่าที่ระยะ $r$ ไกล ๆ จาก dipole:
\begin{equation}
    V(\vb{r})\approx\kem\frac{qd\cos\theta}{r^2}\label{dipolefar}
\end{equation}
และเช่นเดียวกัน quadrupole, octopole, ... จะมีศักย์ที่โตแบบ $1/r^3$, $1/r^4$, ... ตามลำดับ ที่ระยะไกล ๆ

ดังนั้นเราจึงอาจหาวิธีเขียนศักย์ของการกระจายตัวของประจุแบบใด ๆ ให้อยู่ในรูปอนุกรมของพจน์ multipole ($1/r$, $1/r^2$, $1/r^3$, ...) เพื่อที่จะประมาณค่าศักย์ไกล ๆ ด้วยพจน์ monopole และ dipole ได้:

พิจารณาการให้ $\rad$ และ $\alpha$ เป็นมุมและระยะระหว่าง $\vb{r}$ และ $\vb{r}'$ ตามลำดับ จะได้
\[
\rad^2=r^2+(r')^2-2rr'\cos\alpha=r^2\ab(1+\ab(\frac{r'}{r})^2-2\ab(\frac{r'}{r})\cos\alpha)
\]
ดังนั้น
\begin{equation}
    \frac{1}{\rad}=\frac{1}{r}\ab(1+\ab(\frac{r'}{r})\ab(\frac{r'}{r}-2\cos\alpha))^{-1/2}\label{multipoleex}
\end{equation}
จากนั้นใช้ทฤษฎีบททวินามกับ (\ref{multipoleex}) และ (\ref{rodrigues}) จะพิสูจน์ได้ว่า
\begin{equation}
    \frac{1}{\rad}=\frac{1}{r}\sum_{n=0}^\infty\ab(\frac{r'}{r})^nP_n(\cos\alpha)\label{mpradinv}
\end{equation}
นำไปแทนใน (\ref{potentialcont}) ก็จะได้ว่า
\begin{ieqbox}{การกระจาย Multipole}
    V(\vb{r})=\kem\sum_{n=0}^\infty\frac{1}{r^{n+1}}\int(r')^n\,P_n(\cos\alpha)\,\rho(\vb{r}')\odif{\tau'}\label{multipole}
\end{ieqbox}
\subsection{พจน์ Monopole และ Dipole}
สำหรับพจน์ monopole ($n=0$) จะมีค่าเท่ากับ
\[
V_\txt{mon}(\vb{r})=\kem\frac{1}{r}\int P_0(\cos\alpha)\,\rho(\vb{r}')\odif{\tau'}=\kem\frac{1}{r}\int\rho(\vb{r}')\odif{\tau'}
\]
ดังนั้น
\begin{eqbox}{พจน์ Monopole}
    V_\txt{mon}(\vb{r})=\kem\frac{Q}{r}
\end{eqbox}
ซึ่งก็ไม่น่าแปลกใจเพราะค่าศักย์ที่ระยะไกล ๆ ก็ควรจะโตคล้ายประจุรวม $Q$ ในระบบ (เรียก $Q$ นี้ว่า \emph{monopole moment}) โดยพจน์ monopole นี้จะไม่ขึ้นกับตำแหน่งของจุดกำเนิด ต่างจากพจน์อื่น ๆ ที่ขึ้นกับตำแหน่งที่ใช้เป็นจุดกำเนิดในระบบ

ต่อมาพจน์ dipole ($n=1$) จะมีค่าเท่ากับ
\[
V_\txt{dip}(\vb{r})=\kem\frac{1}{r^2}\int r'\,P_1(\cos\alpha)\,\rho(\vb{r}')\odif{\tau'}=\kem\frac{1}{r^2}\int r\cos\alpha\,\rho(\vb{r}')\odif{\tau'}
\]
แต่ว่า $r'\cos\alpha=\vu{r}\cdot\vb{r}'$ ดังนั้น
\[
V_\txt{dip}(\vb{r})=\kem\frac{\vu{r}}{r^2}\cdot\int\vb{r}'\rho(\vb{r}')\odif{\tau'}
\]
อินทิกรัลในด้านขวาไม่ขึ้นกับ $\vb{r}$ ดังนั้นเราจะนิยาม \emph{dipole moment} $\vb{p}$ รอบจุด ๆ หนึ่งว่า:
\begin{defbox}{ Electric Dipole Moment}
    \begin{equation}
        \vb{p}\equiv\int\vb{r}'\rho(\vb{r}')\odif{\tau'}
    \end{equation}
\end{defbox}
ก็จะได้ว่าพจน์ dipole คือ:
\begin{eqbox}{พจน์ Dipole}
    V_\txt{dip}(\vb{r})=\kem\frac{\vb{p}\cdot\vu{r}}{r^2}\label{dippotential}
\end{eqbox}
โดยพจน์ dipole จะไม่ขึ้นกับตำแหน่งของจุดกำเนิดเมื่อประจุรวม $Q=0$ (พิสูจน์จากการแทน $\bar{\vb{r}}=\vb{r}'-\vb{a}$)

\subsection{Dipole บริสุทธิ์}
จาก (\ref{dipolefar}) จะได้ว่า dipole จะเหลือแค่พจน์ dipole ในการกระจาย multipole ถ้าระยะ $\vb{r}$ ไกลมาก ๆ หรืออาจมองกลับกันว่าถ้าระยะ $d$ น้อยมาก ๆ ก็จะเหลือแค่พจน์ dipole เช่นกัน ดังนั้นถ้าเรามองในลิมิต $q\to\infty$ และ $d\to0$ โดยให้ $\vb{p}=q\vb{d}$ คงที่ตลอด จะได้\emph{จุด dipole บริสุทธิ์} ที่จะมีสนามศักย์เป็นเพียง
\begin{equation}
    V(\vb{r})=V_\txt{dip}(\vb{r})=\kem\frac{p\cos\alpha}{r^2}
\end{equation}
ถ้ากำหนดว่า $\vb{p}$ ชี้ในทิศ $+z$ ก็จะได้
\[
V(\vb{r})=\kem\frac{p\cos\theta}{r^2}
\]
ดังนั้นเมื่อใช้ (\ref{gdephi}) จะได้สนามไฟฟ้า:
\begin{align*}
    E_r&=-\pdv{V}{r}=\kem\frac{2p\cos\theta}{r^3}\\
    E_\theta&=-\frac{1}{r}\pdv{V}{\theta}=\kem\frac{p\sin\theta}{r^3}\\
    E_\phi&=-\frac{1}{r\sin\theta}\pdv{V}{\phi}=0
\end{align*}
ดังนั้น
\begin{eqbox}{สนามไฟฟ้าของ Dipole บริสุทธิ์ในพิกัดทรงกลม}
    \vb{E}_\txt{dip}(r,\theta)=\kem\frac{p}{r^3}\ab(2\cos\theta\,\vu{r}+\sin\theta\,\vus{\uptheta})
\end{eqbox}
แสดงว่าสนามไฟฟ้าของ dipole โตแบบ $1/r^3$ (และเช่นเดียวกัน สนามไฟฟ้าของ quadrupole, octopole, ... ก็จะโตแบบ $1/r^4$, $1/r^5$, ... เพราะในการใช้ gradient หาสนามไฟฟ้าจะเพิ่ม $1/r$ ขึ้นมาอีกหนึ่งตัว) แต่สูตรด้านบนยังเป็นสูตรที่ขึ้นกับระบบพิกัดทรงกลม เราสามารถหาสูตรที่ไม่ขึ้นกับระบบพิกัดได้ดังนี้:
\begin{align*}
    \vb{E}_\txt{dip}&=\kem\frac{p}{r^3}\ab(2\cos\theta\,\vu{r}+\sin\theta\,\vus{\uptheta})\\
    &=\kem\frac{1}{r^3}\ab(2p\cos\theta\,\vu{r}+p\sin\theta\,\vus{\uptheta})\\
    &=\kem\frac{1}{r^3}\ab(3p\cos\theta\,\vu{r}+p\sin\theta\,\vus{\uptheta}-p\cos\theta\,\vu{r})\\
    &=\kem\frac{1}{r^3}\ab\Big(3\,\ab(\vb{p}\cdot\vu{r})\,\vu{r}-\vb{p})
\end{align*}
\begin{eqbox}{สนามไฟฟ้าของ Dipole บริสุทธิ์}
    \vb{E}_\txt{dip}=\kem\frac{1}{r^3}\ab\Big(3\,\ab(\vb{p}\cdot\vu{r})\,\vu{r}-\vb{p})\label{dipe}
\end{eqbox}